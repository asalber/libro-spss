\documentclass[a4paper]{article}
\usepackage{svn-multi}
% Version control information:
\svnidlong
{$HeadURL: http://172.20.100.115/svn/maths/practicas_spss/fusion_archivos/fusion_archivos.tex $}
{$LastChangedDate: 2007-10-31 19:03:14 +0100 (mié, 31 oct 2007) $}
{$LastChangedRevision: 2 $}
{$LastChangedBy: alf $}
\svnid{$Id: fusion_archivos.tex 2 2007-10-31 18:03:14Z alf $}
\pdfinfo{/CreationDate (D:\svnpdfdate)}
\svnRegisterAuthor{alf}{Alfredo Sánchez Alberca}

\usepackage[spanish]{babel}
\usepackage[utf8x]{inputenc}
\usepackage{amsmath}
\usepackage{macros}
\usepackage[dvips]{graphicx}
\usepackage{enumitem}
\usepackage{subfigure}
\usepackage[small,bf]{caption2}
\usepackage[top=3cm, bottom=3cm, left=2.54cm, right=2.54cm]{geometry}
\usepackage{fancyhdr}
\pagestyle{fancy}

\lhead{\textsc{Universidad San Pablo CEU}} \rhead{\textsl{\textsf{Departamento de Métodos Cuantitativos}}}
\renewcommand{\headrulewidth}{0pt}
\renewcommand{\floatpagefraction}{.8}
\renewcommand{\textfraction}{.1}
\setcaptionwidth{\textwidth} \addtolength{\captionwidth}{-40pt}
\captionstyle{indent} \setlength\captionindent{\parindent}

\makeatletter
\let\savees@listquot\es@listquot
\def\es@listquot{\protect\savees@listquot}
\makeatletter

\begin{document}
\sloppy

\practica{Práctica de Estadística con SPSS}{Manipulación de datos
y ficheros}

\bigskip
\section*{Ejercicio}

Se ha diseñado un ensayo clínico aleatorizado, doble-ciego y
controlado con placebo, para estudiar el efecto de dos
alternativas terapéuticas en el control de la hipertensión
arterial. Se han reclutado 100 pacientes hipertensos y estos han
sido distribuidos aleatoriamente en tres grupos de tratamiento. A
uno de los grupos (control) se le administró un placebo, a otro
grupo se le administró un inhibidor de la enzima conversora de la
angiotensina (IECA) y al otro un tratamiento combinado de un
diurético y un Antagonista del Calcio. Las variables respuesta
final fueron las presiones arteriales sistólica y diastólica.


Los datos con las claves de aleatorización han sido introducidos
en una base de datos que reside en la central de aleatorización,
mientras que los datos clínicos han sido archivados en dos
archivos distintos, uno para cada uno de los dos centros
participantes en el estudio.

Las variables almacenadas en estos archivos clínicos son las
siguientes:




\begin{itemize}
    \item CLAVE Clave de aleatorización
    \item NOMBRE  Iniciales del paciente
    \item F\_NACIM  Fecha de Nacimiento
    \item F\_INCLUS Fecha de inclusión
    \item SEXO  Sexo (0: Hombre 1: Mujer)
    \item ALTURA Altura en cm.
    \item PESO  Peso en Kg.
    \item PAD\_INI  Presión diastólica basal (inicial)
    \item PAD\_FIN  Presión diastólica final
    \item PAS\_INI Presión sistólica basal (inicial)
    \item PAS\_FIN  Presión sistólica final
\end{itemize}

El archivo de claves de aleatorización contiene sólo dos
variables.

\begin{itemize}
    \item CLAVE Clave de aleatorización
    \item FARMACO Fármaco administrado (0: Placebo, 1: IECA,  2:Ca Antagonista + diurético)
\end{itemize}

Se pide:

\begin{enumerate}
\item Leer los datos del centro con 10 pacientes, incluidos en el
archivo \textsf{Hipertensos HA.xls}, este hospital trabaja con la
hoja de cálculo Excel.

\begin{indicacion}{
Desplegar el menú \menu{Archivo\flecha Abrir\flecha Datos}, e ir hasta el
directorio que contiene el archivo \textsf{Hipertensos HA.xls},
escogiendo como tipo de archivo el formato de Excel. Una vez tenemos
delante el nombre del archivo, para abrirlo será suficiente con un
doble clik, pero teniendo en cuenta que hay que activar la opción
\opcion{Leer nombre de variables} si queremos que utilice la primera
fila del archivo de Excel para dar nombre a las variables en el
archivo de datos de SPSS.}
\end{indicacion}

\item Añadir a estos, los datos de los pacientes 11 al 100,
incluidos en el archivo \textsf{Hipertensos HB} y guardar los datos
como \textsf{Hipertensos totales}.


\begin{indicacion}{
\begin{enumerate}
\item Teniendo en cuenta que lo que pretendemos es añadir nuevos casos
a un archivo de datos ya abierto, el menú a utilizar es
\menu{Datos\flecha Fundir archivos\flecha Añadir casos}. Después activamos la
opción \opcion{Un archivo de datos de SPSS externo}, y con el botón
\boton{Examinar} accedemos hasta la carpeta que contenga el archivo
\textsf{Hipertensos HB}. Una vez seleccionado, utilizamos el botón
\boton{Continuar}, y posteriormente el botón \boton{Aceptar} en el
siguiente cuadro de diálogo que aparece.
\item Una vez generado el archivo de datos, para guardarlo podemos
utilizar el menú \menu{Archivo\flecha Guardar como}.
\end{enumerate}
}
\end{indicacion}

\item Fusionar los datos clínicos con las claves de
aleatorización. El fichero con las clave se denomina
\textsf{Claves aleatorizacion}. Grabar el archivo resultante con
el nombre \textsf{Hipertensos Datos Claves}

\begin{indicacion}{
De forma similar al apartado anterior pero teniendo en cuenta que
ahora lo que pretendemos es añadir variables. Para ello:
\menu{Datos\flecha Fundir archivos\flecha Añadir variables}. Para acceder
después hasta el archivo \textsf{Claves aleatorizacion}, y
posteriormente vamos aceptando en todos los cuadros de diálogo que
aparecen. Una vez generado, guardamos el nuevo archivo de datos y
claves: menú \menu{Archivo\flecha Guardar como}. }
\end{indicacion}

\item Crear un archivo para cada uno de los grupos de tratamiento.
Denominar a estos archivos \textsf{Hipertensos placebo},
\textsf{Hipertensos IECA} e \textsf{Hipertensos Ca}
respectivamente.

\begin{indicacion}{
Se podría lograr el mismo resultado de múltiples maneras. Por
ejemplo:
\begin{enumerate}
\item Segmentando el archivo mediante el menú \menu{Datos\flecha Segmentar
archivo}, y \opcion{Organizar los resultados por grupos} basados en
la variable \variable{farmaco}. Una vez segmentado el archivo,
podemos marcar los casos correspondientes a cada uno de los fármacos
y hacer un cortar y pegar en un nuevo archivo específico para cada
fármaco.
\item También podemos generar los nuevos archivos utilizando el
sistema de filtros de SPSS. Para ello el menú a utilizar es
\menu{Datos\flecha Seleccionar Casos}, con la opción \opcion{Si se
satisface la condición}, botón \boton{Si}, y entramos en un cuadro
de diálogo en el que damos forma a la condición, que en primera
instancia será \variable{farmaco=1}, para volver al cuadro de
diálogo anterior y escoger la opción \opcion{Eliminar casos no
seleccionados}. Con ello nos quedará un archivo de datos que tan
sólo contiene los casos correspondientes al primer fármaco, y
podemos hacer un \menu{Archivo\flecha Guardar como} para ponerle el nombre
adecuado. Igualmente, repetiríamos el proceso para los otros dos
fármacos.
\end{enumerate}}
\end{indicacion}

\item Calcular, para cada paciente, la edad en años el día de la
incorporación al estudio (redondeando al entero más próximo).
Denominar la nueva variable \variable{edad} y etiquetarla
correspondientemente.

\begin{indicacion}{
Internamente SPSS trabaja con las variables en formato fecha
almacenando el número de segundos transcurridos desde el comienzo
del Calendario Gregoriano en el año 1582 hasta la fecha que
introducimos. Por lo tanto, si restamos dos variables en formato
fecha, no nos da el número de años transcurridos entre una y otra,
sino el número de segundos. Por ello, la nueva variable obtenida
como resultado de la resta hay que dividirla entre el número de
segundos que tiene un año a razón de 3600 segundos la hora, 24 horas
el día, y $365.25$ días el año, aproximadamente. Teniendo en cuenta
lo anterior, el proceso a utilizar es: Menú
\menu{Transformar\flecha Calcular variable}, opción \opcion{Variable de
destino}, y damos nombre a la variable \variable{edad}. Como
expresión numérica para su cálculo introducimos:

\[
({\rm F\_INCLUS - F\_NACIM})/(365.25 \times 24 \times 3600)
\]

}
\end{indicacion}

\item Recodificar dicha edad de forma que la nueva variable, de
nombre \variable{grupoedad}, tome los siguientes valores y etiquetas
de valor:

\begin{center}

\begin{tabular}{|l|l|l|}
\hline
\multicolumn{1}{|c|}{Edad en años} & \multicolumn{1}{c|}{Grupo edad} & \multicolumn{1}{c|}{Etiqueta} \\
\hline
\multicolumn{1}{|c|}{Menores de 37} & \multicolumn{1}{c|}{1} & \multicolumn{1}{c|}{Menores de 37} \\
\hline
\multicolumn{1}{|c|}{De 37 a 44} & \multicolumn{1}{c|}{2} & \multicolumn{1}{c|}{De 37 a 44} \\
\hline
\multicolumn{1}{|c|}{De 45 a 52} & \multicolumn{1}{c|}{3} & \multicolumn{1}{c|}{De 45 a 52} \\
\hline
\multicolumn{1}{|c|}{Mayores de 52} & \multicolumn{1}{c|}{4} & \multicolumn{1}{c|}{Mayores de 52} \\
\hline
\end{tabular}

\end{center}

\begin{indicacion}{
Para recodificar una variable, se utiliza el proceso ya explicado en
la práctica de Introducción a SPSS, con el menú
\menu{Transformar\flecha Recodificar en distintas variables}, escogiendo
la \opcion{Variable de entrada}, dando nombre a la \opcion{Variable
de resultado}, utilizando el botón \boton{Cambiar}, y posteriormente
el botón \boton{Valores antiguos y nuevos} para delimitar las
categorías de la nueva variable.

}
\end{indicacion}




\item Calcular, para cada paciente, el índice de masa corporal (se
obtiene dividiendo el peso, expresado en kg, entre la altura,
expresada en m, elevada al cuadrado) y almacenar el resultado en la
variable \variable{masacorp}.

\begin{indicacion}{
Con el menú \menu{Transformar\flecha Calcular variable}, escogiendo como
variable de destino \variable{masacorp}, y como expresión numérica
la indicada en el enunciado.

}
\end{indicacion}

\item Recodificar dicho índice de masa corporal, de forma que la
nueva variable, de nombre \variable{obesidad}, tome los siguientes
valores y etiquetas de valor según el sexo del paciente.

\begin{center}
\begin{tabular}{|l|l|l|l|}
\hline
\multicolumn{1}{|c|}{Sexo} & \multicolumn{1}{c|}{Masacorp} & \multicolumn{1}{c|}{Obesidad} & \multicolumn{1}{c|}{Etiqueta} \\
\hline
\multicolumn{1}{|c|}{} & \multicolumn{1}{c|}{Menor de 21} & \multicolumn{1}{c|}{1} & \multicolumn{1}{c|}{Desnutrido} \\
\cline{2-4}
\multicolumn{1}{|c|}{} & \multicolumn{1}{c|}{De 21,01 a 26,94} & \multicolumn{1}{c|}{2} & \multicolumn{1}{c|}{Normal} \\
\cline{2-4}
\multicolumn{1}{|c|}{0} & \multicolumn{1}{c|}{De 26,95 a 32,94} & \multicolumn{1}{c|}{3} & \multicolumn{1}{c|}{Sobrepeso} \\
\cline{2-4}
\multicolumn{1}{|c|}{} & \multicolumn{1}{c|}{De 32,95 a 43,94} & \multicolumn{1}{c|}{4} & \multicolumn{1}{c|}{Obeso} \\
\cline{2-4}
\multicolumn{1}{|c|}{} & \multicolumn{1}{c|}{Mayor que 43,95} & \multicolumn{1}{c|}{5} & \multicolumn{1}{c|}{Muy obeso} \\
\hline
\multicolumn{1}{|c|}{} & \multicolumn{1}{c|}{Menor de 19} & \multicolumn{1}{c|}{1} & \multicolumn{1}{c|}{Desnutrido} \\
\cline{2-4}
\multicolumn{1}{|c|}{} & \multicolumn{1}{c|}{De 19,01 a 24,94} & \multicolumn{1}{c|}{2} & \multicolumn{1}{c|}{Normal} \\
\cline{2-4}
\multicolumn{1}{|c|}{1} & \multicolumn{1}{c|}{De 24,95 a 29,94} & \multicolumn{1}{c|}{3} & \multicolumn{1}{c|}{Sobrepeso} \\
\cline{2-4}
\multicolumn{1}{|c|}{} & \multicolumn{1}{c|}{De 29,95 a 39,94} & \multicolumn{1}{c|}{4} & \multicolumn{1}{c|}{Obeso} \\
\cline{2-4}
\multicolumn{1}{|c|}{} & \multicolumn{1}{c|}{Mayor que 39,95} & \multicolumn{1}{c|}{5} & \multicolumn{1}{c|}{Muy obeso} \\
\hline
\end{tabular}

\begin{indicacion}{
Se trata de un problema de recodificación, muy parecido al explicado
en la indicación del punto 6, con la única novedad de que ahora hay
que hacer una doble recodificación: por un lado para hombres y por
el otro para mujeres. Para ello, después de escoger las variables de
entrada y de resultado, haciendo uso del botón condicional
\boton{Si} podemos escoger únicamente los casos que cumplen la
condición impuesta por la variable \variable{Sexo}; es decir, con la
condición \variable{Sexo=0}, recodificamos en primera instancia el
índice de masa corporal de los hombres, y con \variable{Sexo=1}
recodificamos el de las mujeres.

}
\end{indicacion}

\end{center}



\end{enumerate}

\end{document}
