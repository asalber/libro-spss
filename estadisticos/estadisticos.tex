% Version control information:
%$HeadURL: https://practicas-spss.googlecode.com/svn/trunk/estadisticos/estadisticos.tex $
%$LastChangedDate: 2010-09-27 16:37:11 +0200 (lun, 27 sep 2010) $
%$LastChangedRevision: 3 $
%$LastChangedBy: asalber $
%$Id: estadisticos.tex 3 2010-09-27 14:37:11Z asalber $

\chapter{Estadísticos Muestrales}

\section{Fundamentos teóricos}

Hemos visto cómo podemos presentar la información que obtenemos de
la muestra, a través de tablas o bien a través de gráficas. La tabla
de frecuencias contiene toda la información de la muestra pero
resulta difícil sacar conclusiones sobre determinados aspectos de la
distribución con sólo mirarla. Ahora veremos cómo a partir de esos
mismos valores observados de la variable estadística, se calculan
ciertos números que resumen la información muestral. Estos números,
llamados \emph{Estadísticos}, se utilizan para poner de manifiesto
ciertos aspectos de la distribución, tales como la dispersión o
concentración de los datos, la forma de su distribución, etc. Según
sea la característica que pretenden reflejar se pueden clasificar en
Medidas de posición, Medidas de dispersión y Medidas de forma.

\subsection{Medidas de posición}

Son valores que indican cómo se sitúan los datos. Los más
importantes son la Media aritmética, la Mediana y la Moda.

\subsubsection{Media aritmética $ \overline{\mbox{\textit{x}}}$}

Se llama \emph{media aritmética} de una variable estadística $X$, y
se representa por $\overline{x}$ , a la suma de todos los resultados
observados, dividida por el tamaño muestral. Es decir, la media de
la variable estadística $X$, cuya distribución de frecuencias
$(x_i,n_i)$, viene dada por

\[\overline{x}=\frac{x_1+\ldots+x_1+\ldots+x_k+\ldots+x_k}{n_1+\ldots+n_k}=\frac{x_1n_1+\ldots+x_kn_k}{n}=\frac{1}{n}\sum_{i=1}^{k}x_in_i
\]

La media aritmética sólo tiene sentido en variables cuantitativas.

\subsubsection{Mediana \textit{Me}}
Se llama \emph{mediana} y lo denotamos por $Me$, a aquel valor de la
muestra que, una vez ordenados todos los valores de la misma en
orden creciente, tiene tantos términos inferiores a él como
superiores. En consecuencia, divide la distribución en dos partes
iguales.

La mediana sólo tiene sentido en atributos ordinales y en
variables cuantitativas.

\subsubsection{Moda \textit{Mo}}
La \emph{moda} es el valor de la variable que presenta una mayor
frecuencia en la muestra. Cuando haya más de un valor con frecuencia
máxima diremos que hay más de una moda. En variables continuas o
discretas agrupadas llamaremos clase modal a la que tenga la máxima
frecuencia. Se puede calcular la moda tanto en variables
cuantitativas como cualitativas.

\subsubsection{Cuantiles}
Si el conjunto total de valores observados se divide en $r$ partes
que contengan cada una $\frac{n}{r}$ observaciones, los puntos de
separación de las mismas reciben el nombre genérico de
\emph{cuantiles}.


Según esto la mediana también es un cuantil con $r=2$.
Algunos cuantiles reciben determinados nombres como:
\begin{description}

\item [Cuartiles.] Son los puntos que dividen la distribución en 4
partes, con igual número de observaciones en cada una de ellas y se 
designan por $C_1,C_2,C_3$. Es claro que $C_2=Me$.

\item[Deciles.] Son los puntos que dividen la distribución en 10
partes, con igual número de observaciones en cada una de ellas y
se designan por $D_1,D_2,\ldots,D_9$.

\item [Percentiles.] Son los puntos que dividen la distribución en
100 partes, con igual número de observaciones en cada una de ellas y
se designan por $P_1,P_2,\ldots,P_{99}$.
\end{description}

\subsection{Medidas de dispersión}
Miden la separación existente entre los valores de la muestra. Las
más importantes son el Rango o Recorrido, el Rango Intercuartílico,
la Varianza, la Desviación Típica y el Coeficiente de Variación.
\subsubsection{Rango o Recorrido \textit{Re}}
La medida de dispersión más inmediata es el rango. Llamamos
\emph{recorrido} o \emph{rango} y lo designaremos por \textit{Re} a
la diferencia entre los valores máximo y mínimo que toma la variable
en la muestra. Es decir

    \[Re = max\{x_i, i=1,2,\ldots,n\} - min\{x_i, i=1,2,\ldots,n\}\]


Este estadístico sirve para medir el campo de variación de la
variable, aunque es la medida de dispersión que menos información
proporciona sobre la mayor o menor agrupación de los valores de la
variable alrededor de las medidas de tendencia central. Además tiene
el inconveniente de que se ve muy afectado por los datos atípicos.

\subsubsection{Rango intercuartílico \textit{RI}}
El \emph{rango intercuartílico} \textit{RI} es la diferencia entre
el tercer y el primer cuartil, y mide, por tanto, el campo de
variación del 50\% de los datos centrales de la distribución. Por
consiguiente
\[ RI=C_3-C_1\]
La ventaja del rango intercuartílico frente al recorrido es que no se ve tan afectado por los datos atípicos.

\subsubsection{Varianza $\textit{s}_\textit{x}^\textrm{2}$}
Llamamos \emph{varianza} de una variable estadística $X$, y la
designaremos por $\textit{s}_\textit{x}^\textrm{2}$, a la media de
los cuadrados de las desviaciones de los valores observados respecto
de la media de la muestra. Así
\[s_x^{2}=\frac{1}{n}\sum_{i=1}^{k}(x_i-\overline{x})^{2}n_i\]

\subsubsection{Desviación típica $\textit{s}_\textit{x}$}
La raíz cuadrada positiva de la varianza se conoce como
\emph{desviación típica} de la variable $X$, y se representa por $s$
\[s=+\sqrt{s_{x}^{2}}\]

\subsubsection{Coeficiente de variación de Pearson $\textit{Cv}_\textit{x}$}
Al cociente entre la desviación típica y el valor absoluto de la
media se le conoce como \emph{coeficiente de variación de Pearson} o
simplemente \emph{coeficiente de variación}:
\[ Cv_x=\frac{s_x}{|\overline{x}|}\]
 El coeficiente de variación es adimensional, y por tanto
permite hacer comparaciones entre variables expresadas en distintas
unidades. Cuanto más próximo esté a 0, menor será la dispersión de
la muestra en relación con la media, y más representativa será ésta
última del conjunto de observaciones.

\subsection{Medidas de forma}
Indican la forma que tiene la distribución de valores en la muestra.
Se pueden clasificar en dos grupos: Medidas de \emph{asimetría}
y medidas de \emph{apuntamiento o curtosis}.

\subsubsection{Coeficiente de asimetría de Fisher $\textit{g}_\textit{1}$}
El \emph{coeficiente de asimetría de Fisher}, que se representa por $g_1$, se define como

\[g_1=\frac{\sum_{i=1}^{k}(x_i-\overline{x})^{3}f_i}{s_x^{3}}\]

Dependiendo del valor que tome tendremos:

\begin{itemize}
  \item  $g_1=0$. Distribución simétrica.
  \item $g_1<0$. Distribución asimétrica hacia la izquierda.
  \item $g_1>0$. Distribución asimétrica hacia la derecha.
\end{itemize}

\subsubsection{Coeficiente de apuntamiento o curtosis $\textit{g}_\textit{2}$}
El grado de apuntamiento de las observaciones de la muestra, se
caracteriza por el \emph{coeficiente de apuntamiento o curtosis} y
se representa por $g_2$

\[g_2=\frac{\sum_{i=1}^{k}(x_i-\overline{x})^{4}f_i}{s_x^{4}}-3\]

Dependiendo del valor que tome tendremos:

\begin{itemize}
  \item $g_2=0$. La distribución tiene un apuntamiento igual que el de la distribución normal de la misma
  media y desviación típica. Se dice que es una distribución \emph{mesocúrtica}.
  \item $g_2<0$. La distribución es menos apuntada que la distribución normal de la misma
  media y desviación típica. Se dice que es una distribución \emph{platicúrtica}.
  \item $g_2>0$. La distribución es más apuntada que la distribución normal de la misma
  media y desviación típica. Se dice que es una distribución \emph{leptocúrtica}.
\end{itemize}

Tanto $g_1$ como $g_2$ suelen utilizarse para comprobar si los datos
muestrales provienen de una población no normal. Cuando $g_1$  está
fuera del intervalo [-2,2] se dice que la distribución es demasiado
asimétrica como para que los datos provengan de una población
normal. Del mismo modo, cuando $g_2$ está fuera del intervalo [-2,2]
se dice que la distribución es, o demasiado apuntada, o demasiado
plana, como para que los datos provengan de una población normal.

\subsection{Estadísticos de variables en las que se definen grupos}
Ya sabemos cómo resumir la información contenida en una muestra
utilizando una serie de estadísticos. Pero hasta ahora sólo hemos
estudiado ejemplos con un único carácter objeto de estudio.

En la mayoría de las investigaciones no estudiaremos un único
carácter, sino un conjunto de caracteres, y muchas veces será
conveniente obtener información de un determinado carácter, en
función de los grupos creados por otro de los caracteres estudiados
en la investigación. A estas variables que se utilizan para formar
grupos se les conoce como \emph{variables clasificadoras} o
\emph{discriminantes}.

Por ejemplo, si se realiza un estudio sobre un conjunto de niños
recién nacidos, podemos estudiar su peso. Pero si además sabemos si
la madre de cada niño es fumadora o no, podremos hacer un estudio
del peso de los niños de las madres fumadoras por un lado y los de
las no fumadoras por otro, para ver si existen diferencias entre
ambos grupos.


\clearpage
\newpage


\section{Ejercicios resueltos}
\begin{enumerate}[leftmargin=*]

\item  Se realizó una encuesta a 40 personas de más de 70
años sobre el número de medicamentos distintos que tomaban
habitualmente. El resultado de dicha encuesta fue el siguiente:
\begin{eqnarray*}
&&3-1-2-2-0-1-4-2-3-5-1-3-2-3-1-4-2-4-3-2 \\
&&3-5-0-1-2-0-2-3-0-1-1-5-3-4-2-3-0-1-2-3
\end{eqnarray*}
Se pide:

\begin{enumerate}
\item  Crear la variable \variable{medicamentos} e introducir los
datos. Si ya se tienen los datos, simplemente recuperarlos.

\item  Calcular la media aritmética, mediana, moda, varianza y
desviación típica de dicha variable. Interpretar los estadísticos.
\begin{indicacion}
\begin{enumerate}
\item Seleccionar el menú \menu{Analizar\flecha Estadísticos
descriptivos\flecha Frecuencias}. \item Seleccionar la variable
\variable{medicamentos} en el campo \opcion{Variables} del cuadro
de diálogo. \item Hacer click sobre el botón \boton{Estadísticos}.
Para seleccionar únicamente los estadísticos que nos piden, marcar
las casillas correspondientes a dichos estadísticos y hacer click
sobre los botones \boton{Continuar} y \boton{Aceptar}.
\end{enumerate}
\end{indicacion}


\item  Calcular el coeficiente de asimetría y el de curtosis e
interpretar los resultados
\begin{indicacion}
Seguir los mismos pasos del apartado anterior, seleccionando ahora los estadísticos que se piden.
\end{indicacion}

\item  Calcular los cuartiles.
\begin{indicacion}
Seguir los mismos pasos de los apartados anteriores, activando la opción \opcion{Cuartiles}.
\end{indicacion}
\end{enumerate}


\item En un grupo de 20 alumnos, las calificaciones obtenidas en
Matemáticas fueron:
\begin{center}
SS - AP - SS - AP - AP - NT - NT - AP - SB - SS \\
SB - SS - AP - AP - NT - AP - SS - NT - SS - NT
\end{center}

Se pide:

\begin{enumerate}
\item  Crear la variable \variable{calificaciones} e introducir
los datos.

\item  Recodificar esta variable, asignando $2.5$ al
SS, $5.5$ al AP, $7.5$ al NT y $9.5$ al SB.
\begin{indicacion}
\begin{enumerate}
\item Seleccionar el menú \menu{Transformar\flecha Recodificar en
distintas variables}.
\item Seleccionar la variable \variable{calificaciones} y hacer click 
sobre el botón con la flecha del cuadro de diálogo para llevarla 
a \variable{Variable de entrada}. 
\item Introducir el nombre de la
nueva variable en el campo \opcion{Nombre} del cuadro de diálogo y
hacer click en el botón \boton{Cambiar}. 
\item Hacer click en el botón \boton{Valores antiguos y nuevos} e 
introducir las reglas de recodificación y hacer click
sobre los botones \boton{Continuar} y \boton{Aceptar}.
\end{enumerate}
\end{indicacion}

\item  Calcular la moda y la mediana.
\begin{indicacion}
\begin{enumerate}
\item Seleccionar el menú \menu{Analizar\flecha Estadísticos
descriptivos\flecha Frecuencias}. \item Seleccionar la variable
recodificada en el campo \opcion{Variables} del cuadro de diálogo.
\item Hacer click sobre el botón \boton{Estadísticos}, seleccionar
los estadísticos que se piden y hacer click sobre los 
botones \boton{Continuar} y \boton{Aceptar}.
\end{enumerate}
\end{indicacion}
\end{enumerate}



\item Para realizar un estudio sobre la estatura de los
estudiantes universitarios, seleccionamos, mediante un proceso de
muestreo aleatorio, una muestra de 30 estudiantes, obteniendo los
siguientes resultados (medidos en centímetros):
\begin{center}
179, 173, 181, 170, 158, 174, 172, 166, 194, 185,\\
162, 187, 198, 177, 178, 165, 154, 188, 166, 171,\\
175, 182, 167, 169, 172, 186, 172, 176, 168, 187.
\end{center}
Se pide:

\begin{enumerate}
\item  Crear la variable \variable{estatura} e introducir los
datos.

\item  Obtener un resumen de estadísticos en el que se muestren la
media aritmética, mediana, moda, varianza, desviación típica y
cuartiles. Interpretar los estadísticos.
\begin{indicacion}
\begin{enumerate}
\item Seleccionar el menú \menu{Analizar\flecha Estadísticos
descriptivos\flecha Frecuencias}. 
\item Seleccionar la variable \variable{estatura} en el 
campo \opcion{Variables} del cuadro de
diálogo. 
\item Hacer click sobre el botón \boton{Estadísticos},
seleccionar los estadísticos que se piden y hacer click sobre los 
botones \boton{Continuar} y \boton{Aceptar}.
\end{enumerate}
\end{indicacion}

\item  Calcular el tercer decil e interpretarlo.
\begin{indicacion}
Seguir los mismos pasos de los apartados anteriores, activando la
opción \opcion{Percentiles} e introduciendo el percentil deseado
en el correspondiente cuadro de texto.}
\end{indicacion}

\item Con los datos obtenidos en apartados anteriores, calcular el
coeficiente de variación de Pearson y el rango intercuartílico, e
interpretar los resultados.
\end{enumerate}

\item Para realizar un estudio sobre la estatura de los
estudiantes universitarios, seleccionamos, mediante un proceso de
muestreo aleatorio, una muestra de 30 estudiantes, obteniendo los
siguientes resultados (medidos en centímetros):

\[
\begin{array}{|c|c|c|c|c|c|}
\hline
   x_i   & $Marca$ & n_i & f_i  & N_i & F_i  \\
\hline
 \mbox{[150,160)} &  155   &  2   & 0,07    & 2   &  0,07 \\
 \mbox{[160,170)} &  165   &  7   & 0,23    & 9   &  0,3 \\
 \mbox{[170,180)} &  175   &  12  & 0,4     & 21  &  0,7 \\
 \mbox{[180,190)} &  185   &  7   & 0,23    & 28  &  0,93 \\
 \mbox{[190,200)} &  195   &  2   & 0,07    & 30  &  1 \\
\hline
\end{array}
\]


Se pide:

\begin{enumerate}

\item  Crear la variable \variable{estatura}, en la que vamos a
introducir las marcas de la clase y crear la variable
\variable{frecuencias}, en la que se introducirán las frecuencias
absolutas.

\item Ponderar los casos de la variable  \variable{estatura} con
las frecuencias de la variable \variable{frecuencias}
\begin{indicacion}
\begin{enumerate}
\item Seleccionar el menú \menu{Datos\flecha Ponderar casos}. \item
Activar la opción \opcion{Ponderar casos mediante}, seleccionar la
variable \variable{frecuencias} y hacer click sobre el botón
\boton{Aceptar}.
\end{enumerate}
\end{indicacion}

\item  Obtener un resumen de estadísticos en el que se muestren la
media aritmética, mediana, moda, varianza, desviación típica y
cuartiles.
\begin{indicacion}
\begin{enumerate}
\item Seleccionar el menú \menu{Analizar\flecha Estadísticos
descriptivos\flecha Frecuencias}. 
\item Seleccionar la variable \variable{estatura} en el 
campo \opcion{Variables} del cuadro de diálogo. 
\item Hacer click sobre el botón \boton{Estadísticos},
seleccionar los estadísticos que se piden y hacer click sobre los 
botones \boton{Continuar} y \boton{Aceptar}.
\end{enumerate}
\end{indicacion}
¿Existen diferencias entre estos estadísticos y los del ejercicio 
anterior? ¿A qué se deben?

\item  Calcular el tercer decil.
\begin{indicacion}
Seguir los mismos pasos de los apartados anteriores, activando la
opción \opcion{Percentiles} e introduciendo el percentil
correspondiente en el cuadro de texto.
\end{indicacion}

\item  Calcular el percentil 62.
\begin{indicacion}
Seguir los pasos de los apartados anteriores seleccionando el estadístico deseado.
\end{indicacion}

\item Con los datos obtenidos en apartados anteriores, calcular el
coeficiente de variación de Pearson y el rango intercuartílico, e
interpretar los resultados.
\end{enumerate}

\item  En un hospital se ha tomado nota de la concentración de
anticuerpos de inmunoglobulina M en el suero sanguíneo de
personas sanas, y han resultado los siguientes datos por litro.
Entre paréntesis figura el sexo de la persona (H para hombre y
M para mujer).
\begin{center}
\begin{tabular}{lllll}
(H) 1.071 & (H) 0.955 & (H) 0.73 & (M) 0.908 & (M) 0.859  \\
(H) 0.927 & (M) 0.962 & (M) 1.543 & (H) 1.094 & (M) 0.847  \\
(H) 1.214 & (M) 1.456 & (M) 1.516 & (M) 1.002 & (M) 0.799  \\
(M) 0.881 & (M) 1.096 & (M) 0.964 & (H) 0.973 & (H) 1.222  \\
(H) 0.887 & (H) 1.022 & (M) 0.881 & (M) 1.42 & (M) 1.205  \\
%      (M) 0.822 & (M) 0.92 & (M) 0.544 & (H) 1.254 & (H) 2.048  \\
%      (M) 1.053 & (M) 0.673 & (M) 1.454 & (H) 1.16 & (H) 1.327  \\
%      (M) 1.005 & (H) 1.017 & (M) 0.806 & (H) 1.337 & (H) 0.926  \\
%      (M) 1.029 & (H) 1.516 & (M) 1.231 & (H) 1.249 & (M) 1.627  \\
%      (M) 1.081 & (H) 1.416 & (M) 1.033 & (M) 1.417 & (M) 1.031  \\
\end{tabular}
\end{center}
Se pide
\begin{enumerate}
\item  Crear las variables \variable{sexo} e \variable{inmunoglobulina} e introducir los datos.
\item  Dividir el archivo, usando como variables de segmentación la variable sexo
\begin{indicacion}
\begin{enumerate}
\item Seleccionar el menú \menu{Datos\flecha Dividir archivo...} 
\item Seleccionar la opción \opcion{comparar los grupos} u
\opcion{Organizar los resultados por grupos} (Se diferencian en la
forma de presentar los resultados).
\item Seleccionar la variable \variable{sexo} en el campo \opcion{Grupos basados en} del cuadro
de diálogo y hacer clic sobre el botón \boton{Aceptar}.
\end{enumerate}
\end{indicacion}

\item  Calcular la media aritmética, la moda y la mediana de la inmunoglobolina, tanto en hombres como en mujeres.
\begin{indicacion}
\begin{enumerate}
\item Seleccionar el menú \menu{Analizar\flecha Estadísticos descriptivos\flecha Frecuencias}.
\item Seleccionar la variable \variable{inmunoglobulina} en el campo \opcion{Variables} del cuadro de diálogo.
\item Hacer click sobre el botón \boton{Estadísticos}, seleccionar los estadísticos que se piden, hacer click sobre el
botón \boton{Continuar} y finalmente hacer click sobre el botón \boton{Aceptar}.
\end{enumerate}
\end{indicacion}

\item  Calcular la varianza y la desviación típica tanto en hombres como en mujeres.
\begin{indicacion}
Seguir los mismos pasos del apartado anterior.
\end{indicacion}

\item  ¿En qué población es más representativa la media, en la de hombres o en la de mujeres?
\begin{indicacion}
Para responder a la pregunta será necesario calcular el coeficiente de variación.
\end{indicacion}
\end{enumerate}

\item Se reciben dos lotes de un determinado fármaco,
fabricados con dos modelos de maquinaria diferentes, además uno
proviene de Madrid y el otro de Valencia, se toma una muestra de
10 cajas, cinco de cada lote y se mide la concentración del
principio activo, obteniendo los siguientes resultados:
\[
\begin{tabular}{|l|c|c|c|c|c|}
  \hline
  Modelo Maquinaria & A & B & A & B & A\\
  \hline
  Procedencia & Madrid & Madrid & Valencia & Madrid & Valencia   \\
  \hline
  Concentraci\'{o}n ($mg/mm^{3}$)  & 17,6 & 19,2 & 21,3 & 15,1 & 17,6\\ 
\hline
\end{tabular}
\]

\[
\begin{tabular}{|l|c|c|c|c|c|}
  \hline
  Modelo Maquinaria & B & A & B & B & A\\
  \hline
  Procedencia & Valencia & Madrid & Valencia & Madrid & Valencia  \\ \hline
  Concentraci\'{o}n ($mg/mm^{3}$) & 18,9 & 16,2 & 18,3 & 19 & 16,4
  \\ \hline
\end{tabular}
\]
Se pide :
\begin{enumerate}
\item Crear las variables \variable{procedencia}, \variable{maquinaria} y \variable{concentracion} e introducir los
datos.

\item Calcular la media aritmética, desviación típica, coeficiente
de asimetría y curtosis de la concentración según el lugar de
procedencia.

\begin{indicacion}
\begin{enumerate}
\item Seleccionar el menú \menu{Datos\flecha Dividir archivo...} 
\item Seleccionar la opción \opcion{comparar los grupos} u \opcion{Organizar los resultados por grupos}.

\item Seleccionar la variable \variable{procedencia} en el campo \opcion{Grupos basados en} del cuadro de diálogo y
hacer clic sobre el botón \boton{Aceptar}. 
\item Seguir los mismos pasos del ejercicio anterior para seleccionar los estadísticos.
\end{enumerate}
\end{indicacion}

\item Dibujar el diagrama de cajas de la concentración de principio activo, según la maquinaria de fabricación.
\begin{indicacion}
\begin{enumerate}
\item Seleccionar el menú \menu{Datos\flecha Dividir archivo...} 
\item Seleccionar la opción \opcion{comparar los grupos} u \opcion{Organizar los resultados por grupos}.
\item Seleccionar la variable \variable{maquinaria} en el campo \opcion{Grupos basados en} del cuadro de diálogo y
hacer clic sobre el botón \boton{Aceptar}.
\item Seleccionar el menú \menu{Gráficos\flecha Cuadros de diálogo antiguos\flecha Diagramas de caja...}.
\item Seleccionar la opción \opcion{Resúmenes para distintas variables} y hacer click en el botón \boton{Definir}. 
\item Seleccionar la variable \variable{concentracion} en el campo \opcion{Las cajas representan} del cuadro de diálogo
y hacer click sobre el botón \boton{Aceptar}.
\end{enumerate}
\end{indicacion}
\end{enumerate}
\end{enumerate}


\section{Ejercicios propuestos}
\begin{enumerate}[leftmargin=*]

\item  El número de lesiones padecidas durante una temporada
por cada jugador de un equipo de fútbol fue el siguiente:
\begin{center}
0 -- 1 -- 2 -- 1 -- 3 -- 0 -- 1 -- 0 -- 1 -- 2 -- 0 -- 1 \\
1 -- 1 -- 2 -- 0 -- 1 -- 3 -- 2 -- 1 -- 2 -- 1 -- 0 -- 1
\end{center}

Se pide:
\begin{enumerate}
  \item Crear la variable lesiones e introducir los datos. Si
ya se tienen los datos, simplemente recuperarlos.
  \item Calcular: media aritmética, mediana, moda, varianza y desviación típica.
  \item Calcular los coeficientes de asimetría y curtosis e interpretar los resultados.
  \item Calcular el cuarto y el octavo decil.
\end{enumerate}



\item En una encuesta sobre la intención de voto en unas
elecciones en las que se presentaban tres partidos $A$, $B$ y $C$,
se preguntó a 30 personas y se obtuvieron las siguientes
respuestas:
\[
\begin{array}{c}
A - B - VB - A - C - A - VB - C - A - A - B - B - A - B - B\\
B - A - A - C - B - B - B - A - VB - A - B - VB - A - B - B
\end{array}
\]

Se pide:

\begin{enumerate}
\item  Crear la variable voto e introducir los datos.

\item  Calcular aquellos estadísticos que sea posible para este
atributo

\end{enumerate}



\item  La siguiente tabla expresa la distribución de las puntuaciones obtenidas por un grupo de alumnos.
\[
\begin{tabular}{|c|c|c|c|c|c|c|c|c|c|}
\hline 0-10 & 10-20 & 20-30 & 30-40 & 40-50 & 50-60 & 60-70 &
70-80 & 80-90 & 90-100
\\ \hline
7 & 8 & 13 & 6 & 7 & 6 & 6 & 5 & 6 & 2 \\ \hline
\end{tabular}
\]
Se pide:

\begin{enumerate}
\item  Calcular la media aritmética, la mediana y la moda.
\item  Calcular el percentil 92.
\item  Calcular la desviación típica.
\item  Calcular el coeficiente de asimetría.
\item  Calcular del coeficiente de curtosis.
\end{enumerate}


\item  En un estudio de población se tomó una muestra de
27 personas, y se les preguntó por su edad y estado civil,
obteniendo los siguientes resultados:
\[\begin{tabular}{|l|c|c|c|c|c|c|c|c|c|}
  \hline
  Estado Civil & Casado & Soltero & Soltero & Viudo & Casado
  & Casado & Divorciado & Soltero & Soltero \\
  \hline
  Edad & 62 & 31 & 45 & 100 & 39 & 62 & 31 & 45 & 100
  \\ \hline
\end{tabular}
\]

\[\begin{tabular}{|l|c|c|c|c|c|c|c|c|c|}
  \hline
  Estado Civil & Soltero & Viudo & Casado
  & Soltero & Divorciado & Viudo & Divorciado & Soltero & Viudo \\ \hline
  Edad & 21 & 38 & 59 & 62 & 65 & 38 & 59 & 62 & 65
  \\ \hline
\end{tabular}
\]

\[\begin{tabular}{|l|c|c|c|c|c|c|c|c|c|}
  \hline
  Estado Civil & Casado & Viudo & Casado
  & Divorciado & Divorciado & Viudo & Viudo & Soltero & Viudo \\ \hline
  Edad & 21 & 31 & 62 & 59 & 65 & 38 & 59 & 31 & 65
  \\ \hline
\end{tabular}
\]

Se pide:
\begin{enumerate}
\item Crear las variables adecuadas e introducir los datos.
\item Calcular la media y la desviación típica de la edad según el estado civil.
\item Dibujar el diagrama de barras para las frecuencias absolutas de la edad según el estado civil.
\end{enumerate}



\item En un estudio se ha medido la tensión arterial de 25 individuos. Además se les ha preguntado si fuman y beben:
\begin{center}
\begin{tabular}{l|ccccccccccccc}
Fumador  & si & no & si & si & si & no & no & si & no & si & no & si & no \\
\hline
Bebedor & no & no & si & si & no & no & si & si & no & si & no & si & si \\
\hline
Tensión arterial & 80 & 92 & 75 & 56 & 89 & 93 & 101 & 67 & 89 & 63 & 98 & 58 & 91 \\
\end{tabular}

\begin{tabular}{l|cccccccccccc}
Fumador  & si & no & no & si & no & no & no & si & no & si & no & si \\
\hline
Bebedor & si & no & si & si & no & no & si & si & si & no & si & no \\
\hline
Tensión arterial & 71 & 52 & 98 & 104 & 57 & 89 & 70 & 93 & 69 & 82 & 70 & 49 \\
\end{tabular}
\end{center}

Se pide :
\begin{enumerate}
\item Crear las variables correspondientes e introducir los datos.

\item Calcular la media aritmética, desviación típica, coeficiente
de asimetría y curtosis de la tensión arterial por grupos
dependiendo de si beben y/o fuman.

\item Dibujar el histograma para las frecuencias absolutas de la
tensión arterial según lo grupos hechos anteriormente.
\end{enumerate}

\end{enumerate}
