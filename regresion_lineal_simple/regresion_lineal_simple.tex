% Version control information:
%$HeadURL: https://practicas-spss.googlecode.com/svn/trunk/regresion_lineal_simple/regresion_lineal_simple.tex $
%$LastChangedDate: 2010-09-27 16:37:11 +0200 (lun, 27 sep 2010) $
%$LastChangedRevision: 3 $
%$LastChangedBy: asalber $
%$Id: regresion_lineal_simple.tex 3 2010-09-27 14:37:11Z asalber $

\chapter{Regresión Lineal Simple}

\section{Fundamentos teóricos}
La \emph{regresión} es la parte de la estadística que trata de determinar la
posible relación entre una variable numérica $Y$, que suele llamarse
\emph{variable dependiente}, y otro conjunto de variables numéricas, $X_1,
X_2,\ldots,X_n$, conocidas como \emph{variables independientes}, de una misma
población. Dicha relación se refleja mediante un modelo funcional
$y=f(x_1,\ldots,x_n)$.

El caso más sencillo se da cuando sólo hay una variable independiente $X$, y
entonces se habla de \emph{regresión simple}. En este caso el modelo que
explica la relación entre $X$ e $Y$ es una función de una variable $y=f(x)$.

Dependiendo de la forma de esta función, existen muchos tipos de regresión
simple. Los más habituales son los que aparecen en la siguiente tabla:
\begin{center}
\begin{tabular}{|l|c|}
\hline
 Familia de curvas       &     Ecuación genérica      \\
\hline\hline
 Lineal                  &          $y=b_0+b_1x$          \\
\hline
 Cuadrática              &       $y=b_0+b_1x+b_2x^2$        \\
\hline
 Cúbica & $y=b_0+b_1x+b_2x^2+b_3x^3$ \\
\hline
 Potencia               &       $y=b_0\cdot x^{b_1}$       \\
\hline
 Exponencial             &     $y=b_0\cdot e^{b_1x}$      \\
\hline
 Logarítmica             &       $y=b_0+b_1\ln x$        \\
\hline
Inversa             &       $y=b_0+\frac{b_1}{x}$        \\
\hline
Compuesto              &       $y=b_0b_1^x$        \\
\hline
Crecimiento             &       $y= e^{b_0 + b_1x}$        \\
\hline
G (Curva-S)             &       $y= e^{b_0 +\frac{b_1}{x} }$        \\
\hline
\end{tabular}
\end{center}

Para elegir un tipo de modelo u otro, se suele representar el \emph{diagrama de
dispersión}, que consiste en dibujar sobre unos ejes cartesianos
correspondientes a las variables $X$ e $Y$, los pares de valores $(x_i,y_j)$
observados en cada individuo de la muestra.

\begin{ejemplo}
En la figura \ref{g:estatura-peso} aparece el diagrama de dispersión
correspondiente a una muestra de 30 individuos en los que se ha medido la estatura
en cm ($X$) y el peso en kg ($Y$). En este caso la forma de la nube
de puntos refleja una relación lineal entre la estatura y el peso.

\begin{figure}[h!]
  \centering
  \scalebox{0.75}{\input{regresion_lineal_simple/img/diagrama_dispersion_estatura_peso}}
  \caption{Diagrama de dispersión. El punto (179,85) indicado corresponde a un
  individuo de la muestra que mide 179 cm y pesa 85 Kg.}\label{g:estatura-peso}
\end{figure}
\end{ejemplo}

Según la forma de la nube de puntos del diagrama, se elige el modelo más
apropiado (figura~\ref{g:tiposrelaciones}), y se determinan los parámetros de dicho modelo para que la función
resultante se ajuste lo mejor posible a la nube de puntos.

\begin{figure}[h!]
\centering 
\subfigure[Sin relación.]{\scalebox{0.5}{\input{regresion_lineal_simple/img/diagrama_dispersion_sin_relacion}}}\qquad
\subfigure[Relación lineal.]{\scalebox{0.5}{\input{regresion_lineal_simple/img/diagrama_dispersion_relacion_lineal}}}\qquad
\subfigure[Relación polinómica.]{\scalebox{0.5}{\input{regresion_lineal_simple/img/diagrama_dispersion_relacion_parabolica}}}\\
\subfigure[Relación exponencial.]{\scalebox{0.5}{\input{regresion_lineal_simple/img/diagrama_dispersion_relacion_exponencial}}}\qquad
\subfigure[Relación logarítmica.]{\scalebox{0.5}{\input{regresion_lineal_simple/img/diagrama_dispersion_relacion_logaritmica}}}\qquad
\subfigure[Relación inversa.]{\scalebox{0.5}{\input{regresion_lineal_simple/img/diagrama_dispersion_relacion_inversa}}}\\
\caption{Diagramas de dispersión correspondientes a distintos tipos de relaciones
entre variables.} \label{g:tiposrelaciones}
\end{figure}

\clearpage

El criterio que suele utilizarse para obtener la función óptima, es que la distancia
de cada punto a la curva, medida en el eje Y, sea lo menor posible. A estas distancias se les
llama \emph{residuos} o \emph{errores} en $Y$ (figura~\ref{g:residuos}). La función
que mejor se ajusta a la nube de puntos será, por tanto, aquella que hace mínima la
suma de los cuadrados de los residuos.\footnote{Se elevan al cuadrado para evitar que en la suma se compensen los residuos positivos con los negativos.}

\begin{figure}[h!]
  \centering
  \scalebox{0.8}{\input{regresion_lineal_simple/img/residuos_y}}
  \caption{Residuos o errores en $Y$. El residuo correspondiente a un punto $(x_i,y_j)$
  es la diferencia entre el valor $y_j$ observado en la muestra, y el valor
  teórico del modelo $f(x_i)$, es decir, $e_{ij}=y_j-f(x_i)$.}\label{g:residuos}
\end{figure}

En el caso de que la nube de puntos tenga forma lineal y optemos por explicar la
relación entre $X$ e $Y$ mediante una recta $y=a+bx$, los parámetros a
determinar son $a$ (punto de corte con el eje de ordenadas) y $b$ (pendiente de
la recta). Los valores de estos parámetros que hacen mínima la suma de
residuos al cuadrado, determinan la recta óptima. Esta recta se conoce como \emph{recta de
regresión de $Y$ sobre $X$} y explica la variable $Y$ en función de la variable
$X$. Su ecuación es
\[ y= \bar{y}+\frac{s_{xy}}{s_x^2}(x-\bar{x}),\]
donde $s_{xy}$ es un estadístico llamado \emph{covarianza} que mide el grado de relación lineal, y cuya fórmula es
\[s_{xy}=\frac{1}{n}\sum_{i,j} (x_i-\bar{x}) (y_j-\bar{y}) n_{ij}.\]

\begin{ejemplo}
En la figura~\ref{g:rectas-estatura-peso} aparecen las rectas de regresión de Estatura sobre Peso y de Peso sobre Estatura del ejemplo anterior.

\begin{figure}[h!]
  \centering
  \scalebox{0.8}{\input{regresion_lineal_simple/img/rectas_regresion}}
  \caption{Rectas de regresión de Estatura sobre Peso y de Peso sobre Estatura. Las rectas de regresión siempre se cortan en el punto de medias $(\bar x, \bar y)$}\label{g:rectas-estatura-peso}
\end{figure}
\end{ejemplo}

La pendiente de la recta de regresión de $Y$ sobre $X$ se conoce como
\emph{coeficiente de regresión de $Y$ sobre $X$}, y mide el incremento que sufrirá
la variable $Y$ por cada unidad que se incremente la variable $X$, según la recta.

Cuanto más pequeños sean los residuos, en valor absoluto, mejor se ajustará el modelo a la nube de
puntos, y por tanto, mejor explicará la relación entre $X$ e $Y$. Cuando todos
los residuos son nulos, la recta pasa por todos los puntos de la nube, y la
relación es perfecta. En este caso ambas rectas, la de $Y$ sobre $X$ y la de
$X$ sobre $Y$ coinciden (figura~\ref{g:dependenciafuncional}).

Por contra, cuando no existe relación lineal entre las variables, la recta de
regresión de $Y$ sobre $X$ tiene pendiente nula, y por tanto la
ecuación es $y=\bar y$, en la que, efectivamente no aparece $x$, o $x=\bar x$
en el caso de la recta de regresión $X$ sobre $Y$, de manera que ambas rectas
se cortan perpendicularmente (figura~\ref{g:independencialineal}).

\begin{figure}[htbp]
\centering 
\subfigure[Dependencia funcional lineal.] {\label{g:dependenciafuncional}
\scalebox{0.7}{\input{regresion_lineal_simple/img/rectas_dependencia_lineal_perfecta}}}\qquad
\subfigure[Independencia lineal.]{\label{g:independencialineal}
\scalebox{0.7}{\input{regresion_lineal_simple/img/rectas_independencia_lineal}}}
\caption{Distintos grados de dependencia. En el primer caso, la relación es perfecta
y los residuos son nulos. En el segundo caso no existe relación lineal y la
pendiente de la recta es nula.}
\end{figure}



\section{Ejercicios prácticos}
\begin{enumerate}[leftmargin=*]
\item Se han medido dos variables $A$ y $B$ en 10 individuos
obteniendo los siguientes resultados:
\begin{center}
\begin{tabular}{c|cccccccccc}
$A$& 0 & 1 & 2 & 3 & 4 & 5 & 6 & 7 & 8 & 9 \\
\hline $B$& 2 & 5 & 8 & 11 & 14 & 17 & 20 & 23 & 26 & 29
\end{tabular}
\end{center}

Se pide:

\begin{enumerate}
\item  Crear las variables \textsf{A} y \textsf{B} e introducir estos datos.


\item  Dibujar el diagrama de dispersión correspondiente.

\begin{indicacion}{
\begin{enumerate}
\item Seleccionar el menú \menu{Gráficos->Dispersión...}, elegir
la opción \texttt{simple} y  hacer click sobre el botón
\texttt{Definir}. \item Seleccionar la variable \textsf{B} en el
campo \texttt{Eje Y} del cuadro de diálogo. \item Seleccionar la
variable \textsf{A} en el campo \texttt{Eje X} del cuadro de
diálogo y hacer click sobre el botón \texttt{Aceptar}.
\end{enumerate}}
\end{indicacion}


En vista del diagrama, ¿qué tipo de modelo crees que explicará
mejor la relación entre el A y B?

\item Calcular la recta de regresión de \textsf{B} sobre
\textsf{A}.

\begin{indicacion}{
\begin{enumerate}
\item Seleccionar el menú \menu{Analizar->Regresión->Lineal...}.

\item Seleccionar la variable \textsf{B} en el campo
\texttt{Dependiente} del cuadro de diálogo.

\item Seleccionar la variable \textsf{A} en el campo
\texttt{Independiente} del cuadro de diálogo y hacer click sobre
el botón \texttt{Aceptar}.

\item Para escribir la recta, observaremos en la ventana de
resultados obtenida, la tabla denominada \texttt{Coeficientes}, y
en la columna \texttt{B} de los \texttt{Coeficientes no
estandarizados}, encontramos en la primera fila la
\textsf{constante} de la recta y en la segunda la
\texttt{pendiente}.

\end{enumerate}}
\end{indicacion}


\item Dibujar dicha recta sobre el diagrama de dispersión.

\begin{indicacion}{
\begin{enumerate}
\item Editar el gráfico realizado anteriormente haciendo un doble
click sobre él.

\item Seleccionar los puntos haciendo click sobre alguno de ellos.
\item Seleccionar el menú \menu{Gráfico->Añadir elemento de
gráfico->Linea de ajuste total} (También se podría usar en lugar
del menu, la barra de herramientas) \item Cerrar el editor de
gráficos, cerrando la ventana.
\end{enumerate}}
\end{indicacion}


\item Calcular la recta de regresión de \textsf{A} sobre
\textsf{B} y dibujarla sobre el correspondiente diagrama de
dispersión.

\begin{indicacion}{
Repetir los pasos de los apartados anteriores pero escogiendo como
variable \texttt{Dependiente} la variable \textsf{A}, y como
variable \texttt{Independiente} la variable \textsf{B}}
\end{indicacion}


\item ¿Son grandes los residuos? Comentar los resultados.
\end{enumerate}


\item  En una licenciatura se quiere estudiar la relación entre el número
medio de horas de estudio diarias y el número de asignaturas suspensas. Para ello se
obtuvo la siguiente muestra:
\begin{center}
\begin{tabular}{cccccccc}
  Horas & Suspensos &  & Horas & Suspensos & & Horas & Suspensos  \\
  \cline{1-2}\cline{4-5}\cline{7-8}
  3.5 & 1 & & 2.2 & 2 & & 1.3 & 4 \\
  0.6 & 5 & & 3.3 & 0 & & 3.1 & 0 \\
  2.8 & 1 & & 1.7 & 3 & & 2.3 & 2 \\
  2.5 & 3 & & 1.1 & 3 & & 3.2 & 2 \\
  2.6 & 1 & & 2.0 & 3 & & 0.9 & 4 \\
  3.9 & 0 & & 3.5 & 0 & & 1.7 & 2 \\
  1.5 & 3 & & 2.1 & 2 & & 0.2 & 5 \\
  0.7 & 3 & & 1.8 & 2 & & 2.9 & 1 \\
  3.6 & 1 & & 1.1 & 4 & & 1.0 & 3 \\
  3.7 & 1 & & 0.7 & 4 & & 2.3 & 2 \\
\end{tabular}

\end{center}

Se pide:

\begin{enumerate}
\item  Crear las variables \textsf{horasestudio} y
\textsf{suspensos} e introducir estos datos.

\item  Calcular la recta de regresión de \textsf{suspensos} sobre
\textsf{horasestudio} y dibujarla.

\begin{indicacion}{
\begin{enumerate}
\item Seleccionar el menú \menu{Analizar->Regresion->Lineal...}.

\item Seleccionar la variable \textsf{suspensos} en el campo
\texttt{Dependiente} del cuadro de diálogo.

\item Seleccionar la variable \textsf{horasestudio} en el campo
\texttt{Independiente} del cuadro de diálogo y hacer click sobre
el botón \texttt{Aceptar}.

\item Para escribir la recta, observaremos en la ventana de
resultados obtenida, la tabla denominada \texttt{Coeficientes}, y
en la columna \texttt{B} de los \texttt{Coeficientes no
estandarizados}, encontramos en la primera fila la
\textsf{constante} de la recta y en la segunda la
\texttt{pendiente}.

\item Seleccionar el menú \menu{Gráficos->Dispersión...}, elegir
la opción \texttt{simple} y  hacer click sobre el botón
\texttt{Definir}. \item Seleccionar la variable \textsf{suspensos}
en el campo \texttt{Eje Y} del cuadro de diálogo. \item
Seleccionar la variable \textsf{horasestudio} en el campo
\texttt{Eje X} del cuadro de diálogo y hacer click sobre el botón
\texttt{Aceptar}.

\item Editar el gráfico realizado anteriormente haciendo un doble
click sobre él.

\item Seleccionar los puntos haciendo click sobre alguno de ellos.
\item Seleccionar el menú \menu{Gráfico->Añadir elemento de
gráfico->Linea de ajuste total} (También se podría usar en lugar
del menu, la barra de herramientas) \item Cerrar el editor de
gráficos, cerrando la ventana.

\end{enumerate}}
\end{indicacion}

\item Indicar el coeficiente de regresión de \textsf{suspensos}
sobre \textsf{horasestudio}. ¿Cómo lo interpretarías?

\item La relación lineal entre estas dos variables, ¿es mejor o peor que la del
ejercicio anterior? Comentar los resultados.
\end{enumerate}
\end{enumerate}

\section{Problemas}
\begin{enumerate}[leftmargin=*]
\item  Se determina la pérdida de actividad que experimenta un
medicamento desde el momento de su fabricación a lo largo del tiempo, obteniéndose
el siguiente resultado:

\begin{center}
\begin{tabular}{|c|c|c|c|c|c|}
\hline Tiempo (en años) & 1 & 2 & 3 & 4 & 5 \\ \hline Actividad restante (\%) & 96 &
84 & 70 & 58 & 52 \\ \hline
\end{tabular}
\end{center}

Se desea calcular:

\begin{enumerate}
\item  La relación fundamental (recta de regresión) entre
actividad restante y tiempo transcurrido.

\item ¿En qué porcentaje disminuye la actividad cada año que pasa?
\end{enumerate}

\item Al realizar un estudio sobre la dosificación de un cierto
medicamento, se trataron 6 pacientes con dosis diarias de 2 mg, 7 pacientes con 3 mg
y otros 7 pacientes con 4 mg. De los pacientes tratados con 2 mg, 2 curaron al cabo
de 5 días, y 4 al cabo de 6 días. De los pacientes tratados con 3 mg diarios, 2
curaron al cabo de 3 días, 4 al cabo de 5 días y 1 al cabo de 6 días. Y de los
pacientes tratados con 4 mg diarios, 5 curaron al cabo de 3 días y 2 al cabo de 5
días. Se pide:

\begin{enumerate}
\item  Calcular la recta de regresión del tiempo de curación con respecto a la dosis
suministrada.

\item  Calcular los coeficientes de regresión. Interpretar los resultados.
\end{enumerate}

\end{enumerate}
