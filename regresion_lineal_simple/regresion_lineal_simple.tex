% Version control information:
%$HeadURL: https://practicas-spss.googlecode.com/svn/trunk/regresion_lineal_simple/regresion_lineal_simple.tex $
%$LastChangedDate: 2010-09-27 16:37:11 +0200 (lun, 27 sep 2010) $
%$LastChangedRevision: 3 $
%$LastChangedBy: asalber $
%$Id: regresion_lineal_simple.tex 3 2010-09-27 14:37:11Z asalber $

\chapter{Regresión Lineal Simple}

\section{Fundamentos teóricos}
La \emph{regresión} es la parte de la estadística que trata de determinar la
posible relación entre una variable numérica $Y$, que suele llamarse
\emph{variable dependiente}, y otro conjunto de variables numéricas, $X_1,
X_2,\ldots,X_n$, conocidas como \emph{variables independientes}, de una misma
población. Dicha relación se refleja mediante un modelo funcional
$y=f(x_1,\ldots,x_n)$.

El caso más sencillo se da cuando sólo hay una variable independiente $X$, y
entonces se habla de \emph{regresión simple}. En este caso el modelo que
explica la relación entre $X$ e $Y$ es una función de una variable $y=f(x)$.

Dependiendo de la forma de esta función, existen muchos tipos de regresión
simple. Los más habituales son los que aparecen en la siguiente tabla:
\begin{center}
\begin{tabular}{|l|c|}
\hline
 Familia de curvas       &     Ecuación genérica      \\
\hline\hline
 Lineal                  &          $y=b_0+b_1x$          \\
\hline
 Cuadrática              &       $y=b_0+b_1x+b_2x^2$        \\
\hline
 Cúbica & $y=b_0+b_1x+b_2x^2+b_3x^3$ \\
\hline
 Potencia               &       $y=b_0\cdot x^{b_1}$       \\
\hline
 Exponencial             &     $y=b_0\cdot e^{b_1x}$      \\
\hline
 Logarítmica             &       $y=b_0+b_1\ln x$        \\
\hline
Inversa             &       $y=b_0+\frac{b_1}{x}$        \\
\hline
Compuesto              &       $y=b_0b_1^x$        \\
\hline
Crecimiento             &       $y= e^{b_0 + b_1x}$        \\
\hline
G (Curva-S)             &       $y= e^{b_0 +\frac{b_1}{x} }$        \\
\hline
\end{tabular}
\end{center}

Para elegir un tipo de modelo u otro, se suele representar el \emph{diagrama de
dispersión}, que consiste en dibujar sobre unos ejes cartesianos
correspondientes a las variables $X$ e $Y$, los pares de valores $(x_i,y_j)$
observados en cada individuo de la muestra.

\begin{ejemplo}
En la figura \ref{g:estatura-peso} aparece el diagrama de dispersión
correspondiente a una muestra de 30 individuos en los que se ha medido la estatura
en cm ($X$) y el peso en kg ($Y$). En este caso la forma de la nube
de puntos refleja una relación lineal entre la estatura y el peso.

\begin{figure}[h!]
  \centering
  \scalebox{0.75}{%% Input file name: diagrama_dispersion_estatura_peso.fig
%% FIG version: 3.2
%% Orientation: Landscape
%% Justification: Flush Left
%% Units: Inches
%% Paper size: A4
%% Magnification: 100.0
%% Resolution: 1200ppi
%% Include the following in the preamble:
%% \usepackage{textcomp}
%% End

\begin{pspicture}(5.94cm,3.48cm)(16.66cm,13.45cm)
\psset{unit=0.8cm}
%%
%% Depth: 2147483647
%%
\newrgbcolor{mycolor0}{1.00 0.50 0.31}\definecolor{mycolor0}{rgb}{1.00,0.50,0.31}
\newgray{mycolor1}{0.74}\definecolor{mycolor1}{gray}{0.74}
%%
%% Depth: 100
%%
\psset{linestyle=solid,linewidth=0.03175,linecolor=mycolor0}
\qdisk(16.02,11.64){0.1}
\qdisk(14.90,8.87){0.1}
\qdisk(16.39,9.70){0.1}
\qdisk(14.34,8.87){0.1}
\qdisk(12.10,6.94){0.1}
\qdisk(15.09,9.01){0.1}
\qdisk(14.71,8.46){0.1}
\qdisk(13.59,8.18){0.1}
\qdisk(18.82,12.33){0.1}
\qdisk(17.14,10.25){0.1}
\qdisk(12.85,7.49){0.1}
\qdisk(17.51,10.67){0.1}
\qdisk(19.56,14.95){0.1}
\qdisk(15.65,8.32){0.1}
\qdisk(15.83,9.56){0.1}
\qdisk(13.41,7.91){0.1}
\qdisk(11.35,6.80){0.1}
\qdisk(16.77,12.74){0.1}
\qdisk(13.59,6.94){0.1}
\qdisk(14.53,8.87){0.1}
\qdisk(15.27,9.56){0.1}
\qdisk(16.58,8.18){0.1}
\qdisk(13.78,8.04){0.1}
\qdisk(14.15,8.46){0.1}
\qdisk(14.71,9.56){0.1}
\qdisk(17.32,9.70){0.1}
\qdisk(14.71,7.35){0.1}
\qdisk(15.46,9.29){0.1}
\qdisk(13.97,9.15){0.1}
\qdisk(17.51,10.95){0.1}
\psset{linecolor=black,fillstyle=none}
\psline(10.61,6.47)(19.94,6.47)
\psline(10.61,6.47)(10.61,6.26)
\psline(12.47,6.47)(12.47,6.26)
\psline(14.34,6.47)(14.34,6.26)
\psline(16.21,6.47)(16.21,6.26)
\psline(18.07,6.47)(18.07,6.26)
\psline(19.94,6.47)(19.94,6.26)
\rput(10.61,5.71){150}
\rput(12.47,5.71){160}
\rput(14.34,5.71){170}
\rput(16.21,5.71){180}
\rput(18.07,5.71){190}
\rput(19.94,5.71){200}
\psline(10.23,6.80)(10.23,15.09)
\psline(10.23,6.80)(10.02,6.80)
\psline(10.23,8.18)(10.02,8.18)
\psline(10.23,9.56)(10.02,9.56)
\psline(10.23,10.95)(10.02,10.95)
\psline(10.23,12.33)(10.02,12.33)
\psline(10.23,13.71)(10.02,13.71)
\psline(10.23,15.09)(10.02,15.09)
\rput{90}(9.73,6.80){50}
\rput{90}(9.73,8.18){60}
\rput{90}(9.73,9.56){70}
\rput{90}(9.73,10.95){80}
\rput{90}(9.73,12.33){90}
\rput{90}(9.73,13.71){100}
\rput{90}(9.73,15.09){110}
\psline(10.23,6.47)(20.31,6.47)(20.31,15.28)(10.23,15.28)(10.23,6.47)
\rput(15.27,15.99){Diagrama de dispersión de Estaturas y Pesos}
\rput(15.27,4.86){Estatura (cm)}
\rput{90}(8.88,10.88){Peso (Kg)}
\psset{linestyle=dashed,linecolor=mycolor1}
\psline(16.02,6.47)(16.02,11.64)
\psline(10.23,11.64)(16.02,11.64)
\rput(16.02,12){$(179,85)$}
\end{pspicture}
%% End
}
  \caption{Diagrama de dispersión. El punto (179,85) indicado corresponde a un
  individuo de la muestra que mide 179 cm y pesa 85 Kg.}\label{g:estatura-peso}
\end{figure}
\end{ejemplo}

Según la forma de la nube de puntos del diagrama, se elige el modelo más
apropiado (figura~\ref{g:tiposrelaciones}), y se determinan los parámetros de dicho modelo para que la función
resultante se ajuste lo mejor posible a la nube de puntos.

\begin{figure}[h!]
\centering 
\subfigure[Sin relación.]{\scalebox{0.5}{%% Input file name: diagrama_dispersion_sin_relacion.fig
%% FIG version: 3.2
%% Orientation: Landscape
%% Justification: Flush Left
%% Units: Inches
%% Paper size: A4
%% Magnification: 100.0
%% Resolution: 1200ppi
%% Include the following in the preamble:
%% \usepackage{textcomp}
%% End

\begin{pspicture}(7.06cm,3.29cm)(16.36cm,13.56cm)
\psset{unit=0.8cm}
%%
%% Depth: 2147483647
%%
\newrgbcolor{mycolor0}{1.00 0.50 0.31}\definecolor{mycolor0}{rgb}{1.00,0.50,0.31}
%%
%% Depth: 100
%%
\psset{linestyle=solid,linewidth=0.03175,linecolor=mycolor0}
\qdisk(18.77,12.67){0.1}
\qdisk(10.15,6.95){0.1}
\qdisk(16.47,7.53){0.1}
\qdisk(16.52,14.13){0.1}
\qdisk(15.03,13.90){0.1}
\qdisk(14.24,9.90){0.1}
\qdisk(16.29,7.29){0.1}
\qdisk(13.52,14.98){0.1}
\qdisk(12.57,8.44){0.1}
\qdisk(16.53,12.51){0.1}
\qdisk(16.20,7.88){0.1}
\qdisk(17.61,8.79){0.1}
\qdisk(11.06,12.08){0.1}
\qdisk(16.83,6.87){0.1}
\qdisk(10.82,11.76){0.1}
\qdisk(13.80,12.41){0.1}
\qdisk(19.55,8.34){0.1}
\qdisk(12.96,12.68){0.1}
\qdisk(16.39,8.39){0.1}
\qdisk(12.97,7.50){0.1}
\qdisk(14.32,11.36){0.1}
\qdisk(16.15,10.56){0.1}
\qdisk(17.20,10.70){0.1}
\qdisk(12.78,14.11){0.1}
\qdisk(17.35,14.54){0.1}
\qdisk(18.21,14.21){0.1}
\qdisk(11.13,9.04){0.1}
\qdisk(15.69,10.55){0.1}
\qdisk(10.89,11.18){0.1}
\qdisk(10.54,13.67){0.1}
\qdisk(17.70,14.68){0.1}
\qdisk(14.67,7.80){0.1}
\qdisk(12.12,15.21){0.1}
\qdisk(19.37,9.75){0.1}
\qdisk(11.19,7.40){0.1}
\qdisk(17.36,13.53){0.1}
\qdisk(14.87,10.39){0.1}
\qdisk(15.55,9.67){0.1}
\qdisk(14.23,13.73){0.1}
\qdisk(12.79,11.91){0.1}
\qdisk(16.15,11.59){0.1}
\qdisk(17.07,10.73){0.1}
\qdisk(16.68,6.09){0.1}
\qdisk(12.60,8.26){0.1}
\qdisk(15.69,10.66){0.1}
\qdisk(16.30,8.04){0.1}
\qdisk(14.04,12.02){0.1}
\qdisk(15.77,14.30){0.1}
\qdisk(12.94,14.05){0.1}
\qdisk(18.20,12.14){0.1}
\qdisk(19.35,6.51){0.1}
\qdisk(11.05,15.07){0.1}
\qdisk(19.06,10.84){0.1}
\qdisk(15.49,12.85){0.1}
\qdisk(14.62,14.33){0.1}
\qdisk(15.09,14.14){0.1}
\qdisk(19.12,6.00){0.1}
\qdisk(15.20,8.40){0.1}
\qdisk(12.89,12.55){0.1}
\qdisk(15.00,7.98){0.1}
\qdisk(17.34,8.76){0.1}
\qdisk(12.23,12.88){0.1}
\qdisk(12.52,7.20){0.1}
\qdisk(15.35,14.11){0.1}
\qdisk(11.78,15.00){0.1}
\qdisk(15.50,6.87){0.1}
\qdisk(14.43,13.71){0.1}
\qdisk(15.10,14.89){0.1}
\qdisk(19.08,12.63){0.1}
\qdisk(11.38,13.02){0.1}
\qdisk(14.59,8.29){0.1}
\qdisk(15.79,11.59){0.1}
\qdisk(17.09,11.97){0.1}
\qdisk(16.88,11.38){0.1}
\qdisk(17.89,6.70){0.1}
\qdisk(13.24,14.50){0.1}
\qdisk(14.26,11.53){0.1}
\qdisk(17.08,5.85){0.1}
\qdisk(11.02,9.44){0.1}
\qdisk(12.08,7.01){0.1}
\qdisk(18.41,7.62){0.1}
\qdisk(17.13,11.49){0.1}
\qdisk(18.09,11.24){0.1}
\qdisk(14.21,11.19){0.1}
\qdisk(13.15,14.78){0.1}
\qdisk(14.75,8.74){0.1}
\qdisk(16.34,15.26){0.1}
\qdisk(12.56,13.18){0.1}
\qdisk(12.58,14.69){0.1}
\qdisk(13.45,8.17){0.1}
\qdisk(15.59,11.11){0.1}
\qdisk(17.54,7.86){0.1}
\qdisk(13.91,9.09){0.1}
\qdisk(12.17,11.26){0.1}
\qdisk(13.35,8.11){0.1}
\qdisk(10.81,8.24){0.1}
\qdisk(19.43,8.62){0.1}
\qdisk(14.42,11.36){0.1}
\qdisk(14.01,10.76){0.1}
\qdisk(14.65,6.91){0.1}
\rput(14.85,16.12){Sin relación}
\rput[l](14.71,4.63){$X$}
\rput[l]{90}(9.35,10.42){$Y$}
\psset{linecolor=black,fillstyle=none}
\psline(9.77,5.48)(19.93,5.48)(19.93,15.64)(9.77,15.64)(9.77,5.48)
\end{pspicture}
%% End
}}\qquad
\subfigure[Relación lineal.]{\scalebox{0.5}{%% Input file name: diagrama_dispersion_relacion_lineal.fig
%% FIG version: 3.2
%% Orientation: Landscape
%% Justification: Flush Left
%% Units: Inches
%% Paper size: A4
%% Magnification: 100.0
%% Resolution: 1200ppi
%% Include the following in the preamble:
%% \usepackage{textcomp}
%% End

\begin{pspicture}(7.06cm,3.29cm)(16.36cm,13.56cm)
\psset{unit=0.8cm}
%%
%% Depth: 2147483647
%%
\newrgbcolor{mycolor0}{1.00 0.50 0.31}\definecolor{mycolor0}{rgb}{1.00,0.50,0.31}
%%
%% Depth: 100
%%
\psset{linestyle=solid,linewidth=0.03175,linecolor=mycolor0}
\qdisk(16.95,12.63){0.1}
\qdisk(18.43,15.26){0.1}
\qdisk(19.26,13.83){0.1}
\qdisk(14.59,10.24){0.1}
\qdisk(18.55,12.88){0.1}
\qdisk(14.48,8.38){0.1}
\qdisk(12.46,9.00){0.1}
\qdisk(14.75,11.02){0.1}
\qdisk(12.80,8.51){0.1}
\qdisk(10.41,7.61){0.1}
\qdisk(18.44,14.54){0.1}
\qdisk(17.23,12.12){0.1}
\qdisk(14.85,10.75){0.1}
\qdisk(13.93,8.79){0.1}
\qdisk(13.42,9.30){0.1}
\qdisk(13.43,9.51){0.1}
\qdisk(17.02,11.54){0.1}
\qdisk(17.17,13.10){0.1}
\qdisk(18.27,12.81){0.1}
\qdisk(14.32,8.29){0.1}
\qdisk(11.13,7.42){0.1}
\qdisk(18.63,13.65){0.1}
\qdisk(11.06,8.06){0.1}
\qdisk(15.23,11.47){0.1}
\qdisk(17.89,12.31){0.1}
\qdisk(18.27,12.17){0.1}
\qdisk(18.73,14.79){0.1}
\qdisk(19.54,14.89){0.1}
\qdisk(12.54,9.33){0.1}
\qdisk(17.19,12.52){0.1}
\qdisk(15.95,11.91){0.1}
\qdisk(18.41,13.01){0.1}
\qdisk(16.70,10.51){0.1}
\qdisk(16.80,11.76){0.1}
\qdisk(10.99,7.56){0.1}
\qdisk(11.26,8.94){0.1}
\qdisk(16.32,13.21){0.1}
\qdisk(12.58,8.97){0.1}
\qdisk(15.02,12.00){0.1}
\qdisk(11.85,7.59){0.1}
\qdisk(18.32,12.93){0.1}
\qdisk(11.07,7.63){0.1}
\qdisk(12.12,9.34){0.1}
\qdisk(18.07,13.64){0.1}
\qdisk(14.07,11.70){0.1}
\qdisk(15.00,9.86){0.1}
\qdisk(18.52,13.72){0.1}
\qdisk(16.90,13.34){0.1}
\qdisk(17.26,11.52){0.1}
\qdisk(10.15,6.52){0.1}
\qdisk(14.05,10.53){0.1}
\qdisk(18.30,13.55){0.1}
\qdisk(11.06,8.80){0.1}
\qdisk(18.12,13.08){0.1}
\qdisk(16.10,12.30){0.1}
\qdisk(10.17,6.67){0.1}
\qdisk(15.01,11.19){0.1}
\qdisk(12.28,8.04){0.1}
\qdisk(11.91,9.12){0.1}
\qdisk(12.05,9.09){0.1}
\qdisk(16.09,11.58){0.1}
\qdisk(15.98,11.74){0.1}
\qdisk(19.55,13.47){0.1}
\qdisk(14.89,10.67){0.1}
\qdisk(16.54,12.68){0.1}
\qdisk(15.80,11.33){0.1}
\qdisk(11.00,7.52){0.1}
\qdisk(14.16,9.83){0.1}
\qdisk(15.05,10.03){0.1}
\qdisk(15.52,11.93){0.1}
\qdisk(16.90,12.77){0.1}
\qdisk(17.52,12.28){0.1}
\qdisk(11.00,6.67){0.1}
\qdisk(12.15,8.79){0.1}
\qdisk(13.90,9.63){0.1}
\qdisk(18.07,12.87){0.1}
\qdisk(11.68,8.49){0.1}
\qdisk(13.01,9.73){0.1}
\qdisk(17.27,11.74){0.1}
\qdisk(17.70,13.12){0.1}
\qdisk(11.27,7.63){0.1}
\qdisk(14.84,11.82){0.1}
\qdisk(13.69,11.74){0.1}
\qdisk(14.07,10.20){0.1}
\qdisk(11.16,7.21){0.1}
\qdisk(15.88,11.23){0.1}
\qdisk(18.83,14.75){0.1}
\qdisk(14.09,10.28){0.1}
\qdisk(11.60,8.57){0.1}
\qdisk(14.04,9.73){0.1}
\qdisk(14.74,10.51){0.1}
\qdisk(10.70,7.36){0.1}
\qdisk(18.75,14.09){0.1}
\qdisk(15.46,10.73){0.1}
\qdisk(13.99,9.88){0.1}
\qdisk(10.73,5.85){0.1}
\qdisk(15.85,9.70){0.1}
\qdisk(11.50,9.04){0.1}
\qdisk(11.36,7.49){0.1}
\qdisk(18.17,13.03){0.1}
\rput(14.85,16.12){Relación lineal}
\rput[l](14.71,4.63){$X$}
\rput[l]{90}(9.35,10.42){$Y$}
\psset{linecolor=black,fillstyle=none}
\psline(9.77,5.48)(19.93,5.48)(19.93,15.64)(9.77,15.64)(9.77,5.48)
\end{pspicture}
%% End
}}\qquad
\subfigure[Relación polinómica.]{\scalebox{0.5}{%% Input file name: diagrama_dispersion_relacion_parabolica.fig
%% FIG version: 3.2
%% Orientation: Landscape
%% Justification: Flush Left
%% Units: Inches
%% Paper size: A4
%% Magnification: 100.0
%% Resolution: 1200ppi
%% Include the following in the preamble:
%% \usepackage{textcomp}
%% End

\begin{pspicture}(7.06cm,3.29cm)(16.36cm,13.56cm)
\psset{unit=0.8cm}
%%
%% Depth: 2147483647
%%
\newrgbcolor{mycolor0}{1.00 0.50 0.31}\definecolor{mycolor0}{rgb}{1.00,0.50,0.31}
%%
%% Depth: 100
%%
\psset{linestyle=solid,linewidth=0.03175,linecolor=mycolor0}
\qdisk(16.95,7.05){0.1}
\qdisk(18.43,10.98){0.1}
\qdisk(19.26,14.20){0.1}
\qdisk(14.59,5.85){0.1}
\qdisk(18.55,15.26){0.1}
\qdisk(14.48,5.89){0.1}
\qdisk(12.46,9.12){0.1}
\qdisk(14.75,5.86){0.1}
\qdisk(12.80,6.85){0.1}
\qdisk(10.41,15.08){0.1}
\qdisk(18.44,13.64){0.1}
\qdisk(17.23,9.07){0.1}
\qdisk(14.85,5.87){0.1}
\qdisk(13.93,6.16){0.1}
\qdisk(13.42,6.71){0.1}
\qdisk(13.43,6.89){0.1}
\qdisk(17.02,8.43){0.1}
\qdisk(17.17,8.85){0.1}
\qdisk(18.27,10.34){0.1}
\qdisk(14.32,5.90){0.1}
\qdisk(11.13,13.56){0.1}
\qdisk(18.63,10.78){0.1}
\qdisk(11.06,11.35){0.1}
\qdisk(15.23,5.87){0.1}
\qdisk(17.89,8.99){0.1}
\qdisk(18.27,11.78){0.1}
\qdisk(18.73,13.44){0.1}
\qdisk(19.54,14.06){0.1}
\qdisk(12.54,8.14){0.1}
\qdisk(17.19,8.27){0.1}
\qdisk(15.95,6.50){0.1}
\qdisk(18.41,11.06){0.1}
\qdisk(16.70,8.24){0.1}
\qdisk(16.80,6.07){0.1}
\qdisk(10.99,10.09){0.1}
\qdisk(11.26,11.93){0.1}
\qdisk(16.32,6.47){0.1}
\qdisk(12.58,7.72){0.1}
\qdisk(15.02,5.89){0.1}
\qdisk(11.85,9.82){0.1}
\qdisk(18.32,9.56){0.1}
\qdisk(11.07,13.06){0.1}
\qdisk(12.12,7.74){0.1}
\qdisk(18.07,9.09){0.1}
\qdisk(14.07,6.14){0.1}
\qdisk(15.00,5.98){0.1}
\qdisk(18.52,11.42){0.1}
\qdisk(16.90,7.06){0.1}
\qdisk(17.26,7.30){0.1}
\qdisk(10.15,13.82){0.1}
\qdisk(14.05,6.81){0.1}
\qdisk(18.30,10.44){0.1}
\qdisk(11.06,12.20){0.1}
\qdisk(18.12,10.97){0.1}
\qdisk(16.10,7.26){0.1}
\qdisk(10.17,13.42){0.1}
\qdisk(15.01,5.85){0.1}
\qdisk(12.28,8.81){0.1}
\qdisk(11.91,9.30){0.1}
\qdisk(12.05,8.93){0.1}
\qdisk(16.09,6.29){0.1}
\qdisk(15.98,6.44){0.1}
\qdisk(19.55,13.43){0.1}
\qdisk(14.89,5.96){0.1}
\qdisk(16.54,7.46){0.1}
\qdisk(15.80,6.47){0.1}
\qdisk(11.00,10.22){0.1}
\qdisk(14.16,5.95){0.1}
\qdisk(15.05,5.86){0.1}
\qdisk(15.52,6.41){0.1}
\qdisk(16.90,8.42){0.1}
\qdisk(17.52,7.97){0.1}
\qdisk(11.00,12.87){0.1}
\qdisk(12.15,8.94){0.1}
\qdisk(13.90,7.48){0.1}
\qdisk(18.07,10.16){0.1}
\qdisk(11.68,8.26){0.1}
\qdisk(13.01,8.48){0.1}
\qdisk(17.27,8.16){0.1}
\qdisk(17.70,10.53){0.1}
\qdisk(11.27,12.32){0.1}
\qdisk(14.84,5.85){0.1}
\qdisk(13.69,6.86){0.1}
\qdisk(14.07,6.32){0.1}
\qdisk(11.16,13.67){0.1}
\qdisk(15.88,6.31){0.1}
\qdisk(18.83,11.79){0.1}
\qdisk(14.09,6.34){0.1}
\qdisk(11.60,9.10){0.1}
\qdisk(14.04,6.02){0.1}
\qdisk(14.74,5.92){0.1}
\qdisk(10.70,13.05){0.1}
\qdisk(18.75,11.38){0.1}
\qdisk(15.46,6.61){0.1}
\qdisk(13.99,6.04){0.1}
\qdisk(10.73,10.42){0.1}
\qdisk(15.85,6.30){0.1}
\qdisk(11.50,12.58){0.1}
\qdisk(11.36,9.86){0.1}
\qdisk(18.17,11.08){0.1}
\rput(14.85,16.12){Relación parabólica}
\rput[l](14.71,4.63){$X$}
\rput[l]{90}(9.35,10.42){$Y$}
\psset{linecolor=black,fillstyle=none}
\psline(9.77,5.48)(19.93,5.48)(19.93,15.64)(9.77,15.64)(9.77,5.48)
\end{pspicture}
%% End
}}\\
\subfigure[Relación exponencial.]{\scalebox{0.5}{%% Input file name: diagrama_dispersion_relacion_exponencial.fig
%% FIG version: 3.2
%% Orientation: Landscape
%% Justification: Flush Left
%% Units: Inches
%% Paper size: A4
%% Magnification: 100.0
%% Resolution: 1200ppi
%% Include the following in the preamble:
%% \usepackage{textcomp}
%% End

\begin{pspicture}(7.06cm,3.29cm)(16.36cm,13.56cm)
\psset{unit=0.8cm}
%%
%% Depth: 2147483647
%%
\newrgbcolor{mycolor0}{1.00 0.50 0.31}\definecolor{mycolor0}{rgb}{1.00,0.50,0.31}
%%
%% Depth: 100
%%
\psset{linestyle=solid,linewidth=0.03175,linecolor=mycolor0}
\qdisk(16.95,8.66){0.1}
\qdisk(18.43,11.23){0.1}
\qdisk(19.26,13.47){0.1}
\qdisk(14.59,6.74){0.1}
\qdisk(18.55,10.19){0.1}
\qdisk(14.48,6.49){0.1}
\qdisk(12.46,7.19){0.1}
\qdisk(14.75,7.06){0.1}
\qdisk(12.80,6.75){0.1}
\qdisk(10.41,6.21){0.1}
\qdisk(18.44,10.17){0.1}
\qdisk(17.23,8.05){0.1}
\qdisk(14.85,7.38){0.1}
\qdisk(13.93,7.62){0.1}
\qdisk(13.42,7.14){0.1}
\qdisk(13.43,6.77){0.1}
\qdisk(17.02,7.22){0.1}
\qdisk(17.17,8.48){0.1}
\qdisk(18.27,9.59){0.1}
\qdisk(14.32,6.61){0.1}
\qdisk(11.13,6.01){0.1}
\qdisk(18.63,9.95){0.1}
\qdisk(11.06,6.06){0.1}
\qdisk(15.23,6.62){0.1}
\qdisk(17.89,9.32){0.1}
\qdisk(18.27,9.78){0.1}
\qdisk(18.73,10.58){0.1}
\qdisk(19.54,15.26){0.1}
\qdisk(12.54,6.34){0.1}
\qdisk(17.19,7.91){0.1}
\qdisk(15.95,7.10){0.1}
\qdisk(18.41,9.80){0.1}
\qdisk(16.70,7.53){0.1}
\qdisk(16.80,8.43){0.1}
\qdisk(10.99,7.07){0.1}
\qdisk(11.26,7.47){0.1}
\qdisk(16.32,7.47){0.1}
\qdisk(12.58,6.68){0.1}
\qdisk(15.02,6.57){0.1}
\qdisk(11.85,7.13){0.1}
\qdisk(18.32,11.28){0.1}
\qdisk(11.07,7.25){0.1}
\qdisk(12.12,6.78){0.1}
\qdisk(18.07,9.43){0.1}
\qdisk(14.07,6.73){0.1}
\qdisk(15.00,7.07){0.1}
\qdisk(18.52,10.64){0.1}
\qdisk(16.90,7.94){0.1}
\qdisk(17.26,8.80){0.1}
\qdisk(10.15,6.25){0.1}
\qdisk(14.05,6.85){0.1}
\qdisk(18.30,10.67){0.1}
\qdisk(11.06,7.26){0.1}
\qdisk(18.12,8.97){0.1}
\qdisk(16.10,7.24){0.1}
\qdisk(10.17,5.85){0.1}
\qdisk(15.01,6.49){0.1}
\qdisk(12.28,6.77){0.1}
\qdisk(11.91,6.84){0.1}
\qdisk(12.05,6.44){0.1}
\qdisk(16.09,7.07){0.1}
\qdisk(15.98,6.97){0.1}
\qdisk(19.55,15.03){0.1}
\qdisk(14.89,6.85){0.1}
\qdisk(16.54,7.48){0.1}
\qdisk(15.80,7.49){0.1}
\qdisk(11.00,6.37){0.1}
\qdisk(14.16,6.58){0.1}
\qdisk(15.05,7.74){0.1}
\qdisk(15.52,7.71){0.1}
\qdisk(16.90,7.84){0.1}
\qdisk(17.52,8.51){0.1}
\qdisk(11.00,6.20){0.1}
\qdisk(12.15,6.05){0.1}
\qdisk(13.90,6.49){0.1}
\qdisk(18.07,9.70){0.1}
\qdisk(11.68,6.94){0.1}
\qdisk(13.01,6.99){0.1}
\qdisk(17.27,8.46){0.1}
\qdisk(17.70,9.07){0.1}
\qdisk(11.27,7.33){0.1}
\qdisk(14.84,7.40){0.1}
\qdisk(13.69,6.25){0.1}
\qdisk(14.07,7.73){0.1}
\qdisk(11.16,6.96){0.1}
\qdisk(15.88,7.00){0.1}
\qdisk(18.83,11.16){0.1}
\qdisk(14.09,6.52){0.1}
\qdisk(11.60,6.04){0.1}
\qdisk(14.04,6.19){0.1}
\qdisk(14.74,6.96){0.1}
\qdisk(10.70,6.17){0.1}
\qdisk(18.75,11.05){0.1}
\qdisk(15.46,6.32){0.1}
\qdisk(13.99,6.06){0.1}
\qdisk(10.73,6.62){0.1}
\qdisk(15.85,7.01){0.1}
\qdisk(11.50,6.76){0.1}
\qdisk(11.36,7.04){0.1}
\qdisk(18.17,9.43){0.1}
\rput(14.85,16.12){Relación exponencial}
\rput[l](14.71,4.63){$X$}
\rput[l]{90}(9.35,10.42){$Y$}
\psset{linecolor=black,fillstyle=none}
\psline(9.77,5.48)(19.93,5.48)(19.93,15.64)(9.77,15.64)(9.77,5.48)
\end{pspicture}
%% End
}}\qquad
\subfigure[Relación logarítmica.]{\scalebox{0.5}{%% Input file name: diagrama_dispersion_relacion_logaritmica.fig
%% FIG version: 3.2
%% Orientation: Landscape
%% Justification: Flush Left
%% Units: Inches
%% Paper size: A4
%% Magnification: 100.0
%% Resolution: 1200ppi
%% Include the following in the preamble:
%% \usepackage{textcomp}
%% End

\begin{pspicture}(7.06cm,3.29cm)(16.36cm,13.56cm)
\psset{unit=0.8cm}
%%
%% Depth: 2147483647
%%
\newrgbcolor{mycolor0}{1.00 0.50 0.31}\definecolor{mycolor0}{rgb}{1.00,0.50,0.31}
%%
%% Depth: 100
%%
\psset{linestyle=solid,linewidth=0.03175,linecolor=mycolor0}
\qdisk(10.25,6.45){0.1}
\qdisk(17.97,14.25){0.1}
\qdisk(10.24,6.13){0.1}
\qdisk(16.74,13.98){0.1}
\qdisk(13.45,11.24){0.1}
\qdisk(11.98,10.24){0.1}
\qdisk(14.03,12.49){0.1}
\qdisk(16.30,12.92){0.1}
\qdisk(12.87,11.39){0.1}
\qdisk(17.66,13.86){0.1}
\qdisk(11.35,9.63){0.1}
\qdisk(16.71,13.56){0.1}
\qdisk(19.53,14.55){0.1}
\qdisk(15.06,12.76){0.1}
\qdisk(15.65,13.66){0.1}
\qdisk(13.18,11.82){0.1}
\qdisk(13.05,11.37){0.1}
\qdisk(16.89,13.60){0.1}
\qdisk(17.06,13.28){0.1}
\qdisk(13.15,11.19){0.1}
\qdisk(15.50,12.48){0.1}
\qdisk(16.41,12.87){0.1}
\qdisk(11.42,9.02){0.1}
\qdisk(15.96,13.66){0.1}
\qdisk(11.49,10.18){0.1}
\qdisk(12.35,9.70){0.1}
\qdisk(19.45,15.26){0.1}
\qdisk(17.93,14.26){0.1}
\qdisk(18.72,14.44){0.1}
\qdisk(17.62,14.22){0.1}
\qdisk(12.71,10.52){0.1}
\qdisk(17.08,14.05){0.1}
\qdisk(15.11,13.01){0.1}
\qdisk(12.50,10.64){0.1}
\qdisk(16.19,13.66){0.1}
\qdisk(15.82,13.18){0.1}
\qdisk(12.90,11.88){0.1}
\qdisk(18.47,14.95){0.1}
\qdisk(12.06,10.13){0.1}
\qdisk(14.62,12.91){0.1}
\qdisk(15.37,12.94){0.1}
\qdisk(10.84,8.18){0.1}
\qdisk(16.16,13.44){0.1}
\qdisk(11.27,8.31){0.1}
\qdisk(14.40,11.97){0.1}
\qdisk(18.34,14.56){0.1}
\qdisk(12.47,10.96){0.1}
\qdisk(16.79,13.98){0.1}
\qdisk(11.01,8.05){0.1}
\qdisk(11.88,10.62){0.1}
\qdisk(17.92,14.09){0.1}
\qdisk(17.86,13.72){0.1}
\qdisk(19.50,14.67){0.1}
\qdisk(11.65,8.98){0.1}
\qdisk(15.13,13.30){0.1}
\qdisk(11.51,8.82){0.1}
\qdisk(16.11,13.46){0.1}
\qdisk(15.11,13.21){0.1}
\qdisk(12.18,11.15){0.1}
\qdisk(10.99,9.12){0.1}
\qdisk(11.28,9.20){0.1}
\qdisk(16.69,13.79){0.1}
\qdisk(12.10,10.21){0.1}
\qdisk(12.22,10.72){0.1}
\qdisk(18.31,14.93){0.1}
\qdisk(17.86,14.36){0.1}
\qdisk(17.02,14.18){0.1}
\qdisk(15.83,13.26){0.1}
\qdisk(12.33,10.39){0.1}
\qdisk(15.52,13.68){0.1}
\qdisk(14.22,12.24){0.1}
\qdisk(14.79,12.44){0.1}
\qdisk(10.71,6.89){0.1}
\qdisk(12.82,10.84){0.1}
\qdisk(12.78,10.72){0.1}
\qdisk(13.79,11.86){0.1}
\qdisk(15.02,12.49){0.1}
\qdisk(18.93,14.52){0.1}
\qdisk(15.15,13.33){0.1}
\qdisk(16.76,14.09){0.1}
\qdisk(10.15,5.85){0.1}
\qdisk(18.14,14.45){0.1}
\qdisk(19.50,14.79){0.1}
\qdisk(19.34,14.92){0.1}
\qdisk(17.47,14.14){0.1}
\qdisk(18.13,14.28){0.1}
\qdisk(12.24,10.52){0.1}
\qdisk(18.14,14.37){0.1}
\qdisk(18.97,14.74){0.1}
\qdisk(15.14,13.44){0.1}
\qdisk(14.97,13.41){0.1}
\qdisk(18.14,14.79){0.1}
\qdisk(17.23,14.34){0.1}
\qdisk(16.82,14.06){0.1}
\qdisk(19.40,14.69){0.1}
\qdisk(19.38,14.66){0.1}
\qdisk(14.72,12.57){0.1}
\qdisk(19.41,14.63){0.1}
\qdisk(16.66,13.91){0.1}
\qdisk(19.55,15.10){0.1}
\rput(14.85,16.12){Relación logarímica}
\rput[l](14.71,4.63){$X$}
\rput[l]{90}(9.35,10.42){$Y$}
\psset{linecolor=black,fillstyle=none}
\psline(9.77,5.48)(19.93,5.48)(19.93,15.64)(9.77,15.64)(9.77,5.48)
\end{pspicture}
%% End
}}\qquad
\subfigure[Relación inversa.]{\scalebox{0.5}{%% Input file name: diagrama_dispersion_relacion_inversa.fig
%% FIG version: 3.2
%% Orientation: Landscape
%% Justification: Flush Left
%% Units: Inches
%% Paper size: A4
%% Magnification: 100.0
%% Resolution: 1200ppi
%% Include the following in the preamble:
%% \usepackage{textcomp}
%% End

\begin{pspicture}(7.06cm,3.29cm)(16.36cm,13.56cm)
\psset{unit=0.8cm}
%%
%% Depth: 2147483647
%%
\newrgbcolor{mycolor0}{1.00 0.50 0.31}\definecolor{mycolor0}{rgb}{1.00,0.50,0.31}
%%
%% Depth: 100
%%
\psset{linestyle=solid,linewidth=0.03175,linecolor=mycolor0}
\qdisk(10.72,9.92){0.1}
\qdisk(11.70,9.02){0.1}
\qdisk(12.50,8.54){0.1}
\qdisk(13.59,7.21){0.1}
\qdisk(15.91,7.19){0.1}
\qdisk(12.47,8.08){0.1}
\qdisk(18.88,6.95){0.1}
\qdisk(13.58,7.44){0.1}
\qdisk(11.98,8.31){0.1}
\qdisk(17.32,6.94){0.1}
\qdisk(18.72,6.19){0.1}
\qdisk(14.93,6.96){0.1}
\qdisk(11.56,9.00){0.1}
\qdisk(15.27,7.49){0.1}
\qdisk(15.63,7.87){0.1}
\qdisk(12.52,8.00){0.1}
\qdisk(11.70,9.08){0.1}
\qdisk(16.82,6.28){0.1}
\qdisk(19.14,5.85){0.1}
\qdisk(10.84,10.82){0.1}
\qdisk(10.92,10.52){0.1}
\qdisk(12.81,7.77){0.1}
\qdisk(12.93,7.81){0.1}
\qdisk(14.78,6.65){0.1}
\qdisk(10.24,15.26){0.1}
\qdisk(10.15,13.68){0.1}
\qdisk(14.47,6.35){0.1}
\qdisk(17.31,6.11){0.1}
\qdisk(17.63,6.60){0.1}
\qdisk(17.37,7.05){0.1}
\qdisk(14.01,7.43){0.1}
\qdisk(15.83,6.39){0.1}
\qdisk(13.08,7.62){0.1}
\qdisk(14.01,7.50){0.1}
\qdisk(17.34,6.06){0.1}
\qdisk(11.58,9.45){0.1}
\qdisk(12.12,8.35){0.1}
\qdisk(13.95,6.41){0.1}
\qdisk(15.85,6.37){0.1}
\qdisk(12.82,7.39){0.1}
\qdisk(15.74,7.45){0.1}
\qdisk(12.18,8.88){0.1}
\qdisk(14.16,7.25){0.1}
\qdisk(15.33,6.41){0.1}
\qdisk(16.55,7.40){0.1}
\qdisk(13.29,8.16){0.1}
\qdisk(13.61,7.80){0.1}
\qdisk(15.76,6.47){0.1}
\qdisk(13.60,7.92){0.1}
\qdisk(10.97,11.36){0.1}
\qdisk(11.15,9.53){0.1}
\qdisk(14.71,6.82){0.1}
\qdisk(14.41,6.13){0.1}
\qdisk(15.64,6.69){0.1}
\qdisk(14.47,7.06){0.1}
\qdisk(17.79,7.19){0.1}
\qdisk(13.02,7.99){0.1}
\qdisk(12.05,7.84){0.1}
\qdisk(16.36,6.09){0.1}
\qdisk(14.64,7.33){0.1}
\qdisk(16.29,6.80){0.1}
\qdisk(12.65,8.67){0.1}
\qdisk(12.30,8.99){0.1}
\qdisk(12.92,8.20){0.1}
\qdisk(11.42,9.83){0.1}
\qdisk(14.74,7.50){0.1}
\qdisk(11.11,9.50){0.1}
\qdisk(17.04,6.90){0.1}
\qdisk(16.72,6.65){0.1}
\qdisk(19.55,6.65){0.1}
\qdisk(18.26,6.51){0.1}
\qdisk(15.61,7.02){0.1}
\qdisk(17.54,6.60){0.1}
\qdisk(17.95,6.72){0.1}
\qdisk(10.44,10.95){0.1}
\qdisk(14.77,8.23){0.1}
\qdisk(12.74,7.76){0.1}
\qdisk(19.49,6.88){0.1}
\qdisk(17.28,6.55){0.1}
\qdisk(13.17,7.79){0.1}
\qdisk(13.52,7.44){0.1}
\qdisk(18.61,6.44){0.1}
\qdisk(13.68,7.60){0.1}
\qdisk(19.05,6.48){0.1}
\qdisk(12.66,8.42){0.1}
\qdisk(17.37,6.01){0.1}
\qdisk(16.33,6.27){0.1}
\qdisk(16.52,6.81){0.1}
\qdisk(15.16,7.02){0.1}
\qdisk(11.12,9.67){0.1}
\qdisk(14.81,7.22){0.1}
\qdisk(10.42,12.66){0.1}
\qdisk(11.34,9.55){0.1}
\qdisk(19.27,6.17){0.1}
\qdisk(11.62,9.62){0.1}
\qdisk(16.55,6.94){0.1}
\qdisk(15.32,6.90){0.1}
\qdisk(15.82,6.83){0.1}
\qdisk(17.60,6.66){0.1}
\qdisk(18.39,6.51){0.1}
\rput(14.85,16.12){Relación inversa}
\rput[l](14.71,4.63){$X$}
\rput[l]{90}(9.35,10.42){$Y$}
\psset{linecolor=black,fillstyle=none}
\psline(9.77,5.48)(19.93,5.48)(19.93,15.64)(9.77,15.64)(9.77,5.48)
\end{pspicture}
%% End
}}\\
\caption{Diagramas de dispersión correspondientes a distintos tipos de relaciones
entre variables.} \label{g:tiposrelaciones}
\end{figure}

\clearpage

El criterio que suele utilizarse para obtener la función óptima, es que la distancia
de cada punto a la curva, medida en el eje Y, sea lo menor posible. A estas distancias se les
llama \emph{residuos} o \emph{errores} en $Y$ (figura~\ref{g:residuos}). La función
que mejor se ajusta a la nube de puntos será, por tanto, aquella que hace mínima la
suma de los cuadrados de los residuos.\footnote{Se elevan al cuadrado para evitar que en la suma se compensen los residuos positivos con los negativos.}

\begin{figure}[h!]
  \centering
  \scalebox{0.8}{%% Input file name: residuos_y.fig
%% FIG version: 3.2
%% Orientation: Landscape
%% Justification: Flush Left
%% Units: Inches
%% Paper size: A4
%% Magnification: 100.0
%% Resolution: 1200ppi

\begin{pspicture}(7.13cm,3.92cm)(16.36cm,13.49cm)
\psset{unit=0.8cm}
%%
%% Depth: 2147483647
%%
\newrgbcolor{mycolor0}{1.00 0.50 0.31}\definecolor{mycolor0}{rgb}{1.00,0.50,0.31}
\newrgbcolor{mycolor1}{0.28 0.46 1.00}\definecolor{mycolor1}{rgb}{0.28,0.46,1.00}
\newgray{mycolor2}{0.74}\definecolor{mycolor2}{gray}{0.74}
%%
%% Depth: 100
%%
\psset{linestyle=solid,linewidth=0.03175,linecolor=mycolor0}
\qdisk(15.82,12.35){0.1}
\qdisk(14.74,9.32){0.1}
\qdisk(16.18,10.23){0.1}
\qdisk(14.21,9.32){0.1}
\qdisk(12.06,7.20){0.1}
\qdisk(14.92,9.47){0.1}
\qdisk(14.57,8.86){0.1}
\qdisk(13.49,8.56){0.1}
\qdisk(18.50,13.11){0.1}
\qdisk(16.89,10.83){0.1}
\qdisk(12.78,7.80){0.1}
\qdisk(17.25,11.29){0.1}
\qdisk(19.22,15.98){0.1}
\qdisk(15.46,8.71){0.1}
\qdisk(15.64,10.08){0.1}
\qdisk(13.32,8.26){0.1}
\qdisk(11.35,7.05){0.1}
\qdisk(17.43,13.56){0.1}
\qdisk(13.49,7.20){0.1}
\qdisk(14.39,9.32){0.1}
\qdisk(15.10,10.08){0.1}
\qdisk(16.36,8.56){0.1}
\qdisk(13.67,8.41){0.1}
\qdisk(14.03,8.86){0.1}
\qdisk(14.57,10.08){0.1}
\qdisk(17.07,10.23){0.1}
\qdisk(14.57,7.65){0.1}
\qdisk(15.28,9.77){0.1}
\qdisk(13.85,9.62){0.1}
\qdisk(17.25,11.59){0.1}
\rput[l](14.96,5.42){$X$}
\rput[l](9,11.37){$Y$}
\psset{linecolor=black,fillstyle=none}
\psline(10.28,6.69)(19.93,6.69)(19.93,16.34)(10.28,16.34)(10.28,6.69)
\psset{linewidth=0.0635}
\psline(11.87,6.69)(19.93,14.36)
\psset{linestyle=dashed,linewidth=0.03175,linecolor=mycolor2}
\psline(17.43,6.69)(17.43,13.56)
\psline(10.28,11.98)(17.43,11.98)
\psline(10.28,13.56)(17.43,13.56)
\psset{linewidth=0.0635, linestyle=solid,linecolor=mycolor1}
\psline{<->}(17.43,11.98)(17.43,13.56)
\rput[r](10.13,11.87){$f(x_i)$}
\rput[t](17.43,6.5){$x_i$}
\rput[r](10,13.53){$y_j$}
\rput[r](16.89,12.62){$e_{ij}=y_j-f(x_i)$}
\rput[l](16.8,13.9){$(x_i,y_j)$}
\end{pspicture}
%% End
}
  \caption{Residuos o errores en $Y$. El residuo correspondiente a un punto $(x_i,y_j)$
  es la diferencia entre el valor $y_j$ observado en la muestra, y el valor
  teórico del modelo $f(x_i)$, es decir, $e_{ij}=y_j-f(x_i)$.}\label{g:residuos}
\end{figure}

En el caso de que la nube de puntos tenga forma lineal y optemos por explicar la
relación entre $X$ e $Y$ mediante una recta $y=a+bx$, los parámetros a
determinar son $a$ (punto de corte con el eje de ordenadas) y $b$ (pendiente de
la recta). Los valores de estos parámetros que hacen mínima la suma de
residuos al cuadrado, determinan la recta óptima. Esta recta se conoce como \emph{recta de
regresión de $Y$ sobre $X$} y explica la variable $Y$ en función de la variable
$X$. Su ecuación es
\[ y= \bar{y}+\frac{s_{xy}}{s_x^2}(x-\bar{x}),\]
donde $s_{xy}$ es un estadístico llamado \emph{covarianza} que mide el grado de relación lineal, y cuya fórmula es
\[s_{xy}=\frac{1}{n}\sum_{i,j} (x_i-\bar{x}) (y_j-\bar{y}) n_{ij}.\]

\begin{ejemplo}
En la figura~\ref{g:rectas-estatura-peso} aparecen las rectas de regresión de Estatura sobre Peso y de Peso sobre Estatura del ejemplo anterior.

\begin{figure}[h!]
  \centering
  \scalebox{0.8}{%% Input file name: rectas_regresion.fig
%% FIG version: 3.2
%% Orientation: Landscape
%% Justification: Flush Left
%% Units: Inches
%% Paper size: A4
%% Magnification: 100.0
%% Resolution: 1200ppi
%% Include the following in the preamble:
%% \usepackage{textcomp}
%% End

\begin{pspicture}(5.94cm,3.48cm)(16.66cm,13.45cm)
\psset{unit=0.8cm}
%%
%% Depth: 2147483647
%%
\newrgbcolor{mycolor0}{1.00 0.50 0.31}\definecolor{mycolor0}{rgb}{1.00,0.50,0.31}
\newrgbcolor{mycolor1}{0.28 0.46 1.00}\definecolor{mycolor1}{rgb}{0.28,0.46,1.00}
%%
%% Depth: 100
%%
\psset{linestyle=solid,linewidth=0.03175,linecolor=mycolor0}
\qdisk(16.02,11.64){0.1}
\qdisk(14.90,8.87){0.1}
\qdisk(16.39,9.70){0.1}
\qdisk(14.34,8.87){0.1}
\qdisk(12.10,6.94){0.1}
\qdisk(15.09,9.01){0.1}
\qdisk(14.71,8.46){0.1}
\qdisk(13.59,8.18){0.1}
\qdisk(18.82,12.33){0.1}
\qdisk(17.14,10.25){0.1}
\qdisk(12.85,7.49){0.1}
\qdisk(17.51,10.67){0.1}
\qdisk(19.56,14.95){0.1}
\qdisk(15.65,8.32){0.1}
\qdisk(15.83,9.56){0.1}
\qdisk(13.41,7.91){0.1}
\qdisk(11.35,6.80){0.1}
\qdisk(17.70,12.74){0.1}
\qdisk(13.59,6.94){0.1}
\qdisk(14.53,8.87){0.1}
\qdisk(15.27,9.56){0.1}
\qdisk(16.58,8.18){0.1}
\qdisk(13.78,8.04){0.1}
\qdisk(14.15,8.46){0.1}
\qdisk(14.71,9.56){0.1}
\qdisk(17.32,9.70){0.1}
\qdisk(14.71,7.35){0.1}
\qdisk(15.46,9.29){0.1}
\qdisk(13.97,9.15){0.1}
\qdisk(17.51,10.95){0.1}
\psset{linecolor=black,fillstyle=none}
\psline(10.61,6.47)(19.94,6.47)
\psline(10.61,6.47)(10.61,6.26)
\psline(12.47,6.47)(12.47,6.26)
\psline(14.34,6.47)(14.34,6.26)
\psline(16.21,6.47)(16.21,6.26)
\psline(18.07,6.47)(18.07,6.26)
\psline(19.94,6.47)(19.94,6.26)
\rput(10.61,5.71){150}
\rput(12.47,5.71){160}
\rput(14.34,5.71){170}
\rput(16.21,5.71){180}
\rput(18.07,5.71){190}
\rput(19.94,5.71){200}
\psline(10.23,6.80)(10.23,15.09)
\psline(10.23,6.80)(10.02,6.80)
\psline(10.23,8.18)(10.02,8.18)
\psline(10.23,9.56)(10.02,9.56)
\psline(10.23,10.95)(10.02,10.95)
\psline(10.23,12.33)(10.02,12.33)
\psline(10.23,13.71)(10.02,13.71)
\psline(10.23,15.09)(10.02,15.09)
\rput{90}(9.73,6.80){50}
\rput{90}(9.73,8.18){60}
\rput{90}(9.73,9.56){70}
\rput{90}(9.73,10.95){80}
\rput{90}(9.73,12.33){90}
\rput{90}(9.73,13.71){100}
\rput{90}(9.73,15.09){110}
\psline(10.23,6.47)(20.31,6.47)(20.31,15.28)(10.23,15.28)(10.23,6.47)
\rput(15.27,15.99){Rectas de regresión entre Estaturas y Pesos}
\rput(15.27,4.86){Estatura (cm)}
\rput{90}(8.88,10.88){Peso (Kg)}
\psset{linewidth=0.0635}
\psline(11.18,6.47)(20.31,13.37)
\psline(12.62,6.47)(20.11,15.28)
\psset{linewidth=0.03175,linecolor=mycolor1}
\qdisk(15.21,9.52){0.1}
\rput(14.9,10){$(\bar x,\bar y)$}
\rput[r](18.5,13.62){Estatura sobre Peso}
\rput[l](18,11){Peso sobre }
\rput[l](18,10.5){Estatura}
\end{pspicture}
%% End
}
  \caption{Rectas de regresión de Estatura sobre Peso y de Peso sobre Estatura. Las rectas de regresión siempre se cortan en el punto de medias $(\bar x, \bar y)$}\label{g:rectas-estatura-peso}
\end{figure}
\end{ejemplo}

La pendiente de la recta de regresión de $Y$ sobre $X$ se conoce como
\emph{coeficiente de regresión de $Y$ sobre $X$}, y mide el incremento que sufrirá
la variable $Y$ por cada unidad que se incremente la variable $X$, según la recta.

Cuanto más pequeños sean los residuos, en valor absoluto, mejor se ajustará el modelo a la nube de
puntos, y por tanto, mejor explicará la relación entre $X$ e $Y$. Cuando todos
los residuos son nulos, la recta pasa por todos los puntos de la nube, y la
relación es perfecta. En este caso ambas rectas, la de $Y$ sobre $X$ y la de
$X$ sobre $Y$ coinciden (figura~\ref{g:dependenciafuncional}).

Por contra, cuando no existe relación lineal entre las variables, la recta de
regresión de $Y$ sobre $X$ tiene pendiente nula, y por tanto la
ecuación es $y=\bar y$, en la que, efectivamente no aparece $x$, o $x=\bar x$
en el caso de la recta de regresión $X$ sobre $Y$, de manera que ambas rectas
se cortan perpendicularmente (figura~\ref{g:independencialineal}).

\begin{figure}[htbp]
\centering 
\subfigure[Dependencia funcional lineal.] {\label{g:dependenciafuncional}
\scalebox{0.7}{%% Input file name: rectas_dependencia_lineal_perfecta.fig
%% FIG version: 3.2
%% Orientation: Landscape
%% Justification: Flush Left
%% Units: Inches
%% Paper size: A4
%% Magnification: 100.0
%% Resolution: 1200ppi
%% Include the following in the preamble:
%% \usepackage{textcomp}
%% End

\begin{pspicture}(7.06cm,3.29cm)(16.36cm,13.56cm)
\psset{unit=0.8cm}
%%
%% Depth: 2147483647
%%
\newrgbcolor{mycolor0}{1.00 0.50 0.31}\definecolor{mycolor0}{rgb}{1.00,0.50,0.31}
%%
%% Depth: 100
%%
\psset{linestyle=solid,linewidth=0.03175,linecolor=mycolor0}
\qdisk(11.95,8.54){0.1}
\qdisk(11.84,8.46){0.1}
\qdisk(19.55,13.83){0.1}
\qdisk(16.48,11.69){0.1}
\qdisk(17.77,12.59){0.1}
\qdisk(12.99,9.26){0.1}
\qdisk(15.08,10.72){0.1}
\qdisk(14.45,10.28){0.1}
\qdisk(17.63,12.49){0.1}
\qdisk(19.36,13.70){0.1}
\qdisk(12.36,8.82){0.1}
\qdisk(14.07,10.02){0.1}
\qdisk(14.87,10.57){0.1}
\qdisk(16.37,11.62){0.1}
\qdisk(10.49,7.52){0.1}
\qdisk(19.23,13.61){0.1}
\qdisk(13.17,9.39){0.1}
\qdisk(16.36,11.61){0.1}
\qdisk(18.39,13.03){0.1}
\qdisk(18.61,13.18){0.1}
\qdisk(18.99,13.44){0.1}
\qdisk(15.54,11.04){0.1}
\qdisk(19.25,13.62){0.1}
\qdisk(17.47,12.38){0.1}
\qdisk(13.14,9.37){0.1}
\qdisk(18.57,13.15){0.1}
\qdisk(16.59,11.77){0.1}
\qdisk(18.99,13.44){0.1}
\qdisk(19.05,13.48){0.1}
\qdisk(19.34,13.68){0.1}
\qdisk(17.73,12.56){0.1}
\qdisk(12.21,8.72){0.1}
\qdisk(19.09,13.51){0.1}
\qdisk(19.53,13.82){0.1}
\qdisk(10.50,7.53){0.1}
\qdisk(13.27,9.46){0.1}
\qdisk(12.90,9.20){0.1}
\qdisk(10.18,7.30){0.1}
\qdisk(10.15,7.28){0.1}
\qdisk(14.76,10.49){0.1}
\qdisk(15.92,11.31){0.1}
\qdisk(17.66,12.52){0.1}
\qdisk(18.46,13.07){0.1}
\qdisk(14.35,10.21){0.1}
\qdisk(13.63,9.71){0.1}
\qdisk(10.32,7.41){0.1}
\qdisk(19.36,13.70){0.1}
\qdisk(14.73,10.48){0.1}
\qdisk(12.94,9.23){0.1}
\qdisk(13.98,9.95){0.1}
\qdisk(19.55,13.83){0.1}
\qdisk(14.46,10.29){0.1}
\qdisk(14.41,10.25){0.1}
\qdisk(17.91,12.69){0.1}
\qdisk(14.87,10.57){0.1}
\qdisk(12.97,9.25){0.1}
\qdisk(14.48,10.30){0.1}
\qdisk(15.61,11.08){0.1}
\qdisk(18.44,13.06){0.1}
\qdisk(13.92,9.91){0.1}
\qdisk(10.90,7.81){0.1}
\qdisk(16.54,11.73){0.1}
\qdisk(10.82,7.75){0.1}
\qdisk(11.73,8.38){0.1}
\qdisk(12.02,8.59){0.1}
\qdisk(13.27,9.46){0.1}
\qdisk(13.20,9.41){0.1}
\qdisk(17.86,12.65){0.1}
\qdisk(17.05,12.09){0.1}
\qdisk(14.88,10.58){0.1}
\qdisk(16.77,11.89){0.1}
\qdisk(14.26,10.15){0.1}
\qdisk(11.68,8.35){0.1}
\qdisk(13.77,9.81){0.1}
\qdisk(10.45,7.49){0.1}
\qdisk(16.97,12.03){0.1}
\qdisk(11.75,8.40){0.1}
\qdisk(17.22,12.21){0.1}
\qdisk(13.91,9.91){0.1}
\qdisk(18.12,12.84){0.1}
\qdisk(19.39,13.72){0.1}
\qdisk(14.97,10.64){0.1}
\qdisk(12.90,9.20){0.1}
\qdisk(13.09,9.33){0.1}
\qdisk(11.98,8.56){0.1}
\qdisk(14.01,9.98){0.1}
\qdisk(15.78,11.21){0.1}
\qdisk(15.26,10.85){0.1}
\qdisk(14.42,10.26){0.1}
\qdisk(11.24,8.05){0.1}
\qdisk(15.58,11.07){0.1}
\qdisk(11.99,8.57){0.1}
\qdisk(10.24,7.35){0.1}
\qdisk(16.92,12.00){0.1}
\qdisk(17.51,12.41){0.1}
\qdisk(15.24,10.83){0.1}
\qdisk(11.73,8.38){0.1}
\qdisk(15.33,10.89){0.1}
\qdisk(14.29,10.17){0.1}
\qdisk(11.55,8.26){0.1}
\rput(14.85,16.12){Relación lineal perfecta}
\rput[l](14.71,4.63){$X$}
\rput[l]{90}(9.35,10.42){$Y$}
\psset{linecolor=black,fillstyle=none}
\psline(9.77,5.48)(19.93,5.48)(19.93,15.64)(9.77,15.64)(9.77,5.48)
\psset{linewidth=0.0635}
\psline(9.77,7.02)(19.93,14.10)
\rput[l](14.93,9.49){$X$ sobre $Y$ $=$ $Y$ sobre $X$}
\end{pspicture}
%% End
}}\qquad
\subfigure[Independencia lineal.]{\label{g:independencialineal}
\scalebox{0.7}{%% Input file name: rectas_independencia_lineal.fig
%% FIG version: 3.2
%% Orientation: Landscape
%% Justification: Flush Left
%% Units: Inches
%% Paper size: A4
%% Magnification: 100.0
%% Resolution: 1200ppi
%% Include the following in the preamble:
%% \usepackage{textcomp}
%% End

\begin{pspicture}(7.06cm,3.29cm)(16.36cm,13.56cm)
\psset{unit=0.8cm}
%%
%% Depth: 2147483647
%%
\newrgbcolor{mycolor0}{1.00 0.50 0.31}\definecolor{mycolor0}{rgb}{1.00,0.50,0.31}
%%
%% Depth: 100
%%
\psset{linestyle=solid,linewidth=0.03175,linecolor=mycolor0}
\qdisk(16.63,9.96){0.1}
\qdisk(18.53,10.92){0.1}
\qdisk(13.60,12.64){0.1}
\qdisk(12.83,13.91){0.1}
\qdisk(16.21,6.16){0.1}
\qdisk(16.04,9.07){0.1}
\qdisk(17.89,6.74){0.1}
\qdisk(13.39,11.97){0.1}
\qdisk(13.66,9.79){0.1}
\qdisk(18.77,11.43){0.1}
\qdisk(18.79,10.20){0.1}
\qdisk(15.90,13.76){0.1}
\qdisk(16.33,11.13){0.1}
\qdisk(18.47,9.84){0.1}
\qdisk(11.42,6.20){0.1}
\qdisk(10.15,9.35){0.1}
\qdisk(15.48,12.85){0.1}
\qdisk(12.40,9.99){0.1}
\qdisk(13.81,13.59){0.1}
\qdisk(17.18,6.11){0.1}
\qdisk(17.94,8.41){0.1}
\qdisk(17.76,13.78){0.1}
\qdisk(13.97,8.58){0.1}
\qdisk(18.60,5.92){0.1}
\qdisk(10.92,14.58){0.1}
\qdisk(16.44,12.66){0.1}
\qdisk(18.74,14.83){0.1}
\qdisk(12.98,11.21){0.1}
\qdisk(10.70,8.89){0.1}
\qdisk(10.33,7.27){0.1}
\qdisk(17.96,11.10){0.1}
\qdisk(12.36,15.04){0.1}
\qdisk(17.98,13.34){0.1}
\qdisk(19.55,8.17){0.1}
\qdisk(10.25,6.88){0.1}
\qdisk(12.06,15.26){0.1}
\qdisk(15.21,8.56){0.1}
\qdisk(11.72,8.30){0.1}
\qdisk(12.25,10.12){0.1}
\qdisk(12.58,12.44){0.1}
\qdisk(11.91,12.42){0.1}
\qdisk(14.86,11.21){0.1}
\qdisk(13.75,13.21){0.1}
\qdisk(15.09,15.10){0.1}
\qdisk(12.35,7.22){0.1}
\qdisk(10.73,9.81){0.1}
\qdisk(10.43,14.15){0.1}
\qdisk(11.32,8.59){0.1}
\qdisk(16.15,12.81){0.1}
\qdisk(19.51,7.91){0.1}
\qdisk(11.44,13.64){0.1}
\qdisk(11.21,12.44){0.1}
\qdisk(18.86,6.46){0.1}
\qdisk(10.31,10.08){0.1}
\qdisk(14.02,13.31){0.1}
\qdisk(17.26,8.80){0.1}
\qdisk(13.32,8.87){0.1}
\qdisk(11.63,6.47){0.1}
\qdisk(18.64,9.07){0.1}
\qdisk(11.00,7.62){0.1}
\qdisk(11.52,5.85){0.1}
\qdisk(15.30,7.81){0.1}
\qdisk(10.55,6.39){0.1}
\qdisk(17.38,12.21){0.1}
\qdisk(15.61,13.11){0.1}
\qdisk(10.75,12.83){0.1}
\qdisk(12.50,8.51){0.1}
\qdisk(17.62,13.47){0.1}
\qdisk(12.53,11.83){0.1}
\qdisk(12.42,6.73){0.1}
\qdisk(16.71,12.34){0.1}
\qdisk(13.96,13.40){0.1}
\qdisk(12.70,14.84){0.1}
\qdisk(18.84,8.00){0.1}
\qdisk(10.69,9.75){0.1}
\qdisk(14.34,8.62){0.1}
\qdisk(18.00,11.21){0.1}
\qdisk(14.23,6.39){0.1}
\qdisk(11.13,8.85){0.1}
\qdisk(16.74,7.02){0.1}
\qdisk(13.69,13.22){0.1}
\qdisk(11.43,10.96){0.1}
\qdisk(14.30,6.05){0.1}
\qdisk(18.77,13.78){0.1}
\qdisk(12.15,6.08){0.1}
\qdisk(10.41,12.01){0.1}
\qdisk(16.28,10.93){0.1}
\qdisk(19.51,12.10){0.1}
\qdisk(10.27,7.83){0.1}
\qdisk(17.47,8.75){0.1}
\qdisk(15.26,9.19){0.1}
\qdisk(17.82,6.24){0.1}
\qdisk(16.99,15.21){0.1}
\qdisk(10.35,7.39){0.1}
\qdisk(15.29,8.17){0.1}
\qdisk(19.06,6.05){0.1}
\qdisk(17.96,8.48){0.1}
\qdisk(15.95,8.06){0.1}
\qdisk(15.13,6.57){0.1}
\qdisk(14.13,10.01){0.1}
\rput(14.85,16.12){Sin relación lineal}
\rput[l](14.71,4.63){$X$}
\rput[l]{90}(9,10.42){$Y$}
\psset{linecolor=black,fillstyle=none}
\psline(9.77,5.48)(19.93,5.48)(19.93,15.64)(9.77,15.64)(9.77,5.48)
\psset{linewidth=0.0635}
\psline(9.77,10.16)(19.93,10.16)
\rput[r](9.5,10.16){$\bar y$}
\psline(14.57,5.48)(14.57,15.64)
\rput[t](14.57,5.2){$\bar x$}
\rput[l](14.72,6.85){$X$ sobre $Y$}
\rput[l](17.82,9.57){$Y$ sobre $X$}
\end{pspicture}
%% End
}}
\caption{Distintos grados de dependencia. En el primer caso, la relación es perfecta
y los residuos son nulos. En el segundo caso no existe relación lineal y la
pendiente de la recta es nula.}
\end{figure}



\section{Ejercicios prácticos}
\begin{enumerate}[leftmargin=*]
\item Se han medido dos variables $A$ y $B$ en 10 individuos
obteniendo los siguientes resultados:
\begin{center}
\begin{tabular}{c|cccccccccc}
$A$& 0 & 1 & 2 & 3 & 4 & 5 & 6 & 7 & 8 & 9 \\
\hline $B$& 2 & 5 & 8 & 11 & 14 & 17 & 20 & 23 & 26 & 29
\end{tabular}
\end{center}

Se pide:

\begin{enumerate}
\item  Crear las variables \textsf{A} y \textsf{B} e introducir estos datos.


\item  Dibujar el diagrama de dispersión correspondiente.

\begin{indicacion}{
\begin{enumerate}
\item Seleccionar el menú \menu{Gráficos->Dispersión...}, elegir
la opción \texttt{simple} y  hacer click sobre el botón
\texttt{Definir}. \item Seleccionar la variable \textsf{B} en el
campo \texttt{Eje Y} del cuadro de diálogo. \item Seleccionar la
variable \textsf{A} en el campo \texttt{Eje X} del cuadro de
diálogo y hacer click sobre el botón \texttt{Aceptar}.
\end{enumerate}}
\end{indicacion}


En vista del diagrama, ¿qué tipo de modelo crees que explicará
mejor la relación entre el A y B?

\item Calcular la recta de regresión de \textsf{B} sobre
\textsf{A}.

\begin{indicacion}{
\begin{enumerate}
\item Seleccionar el menú \menu{Analizar->Regresión->Lineal...}.

\item Seleccionar la variable \textsf{B} en el campo
\texttt{Dependiente} del cuadro de diálogo.

\item Seleccionar la variable \textsf{A} en el campo
\texttt{Independiente} del cuadro de diálogo y hacer click sobre
el botón \texttt{Aceptar}.

\item Para escribir la recta, observaremos en la ventana de
resultados obtenida, la tabla denominada \texttt{Coeficientes}, y
en la columna \texttt{B} de los \texttt{Coeficientes no
estandarizados}, encontramos en la primera fila la
\textsf{constante} de la recta y en la segunda la
\texttt{pendiente}.

\end{enumerate}}
\end{indicacion}


\item Dibujar dicha recta sobre el diagrama de dispersión.

\begin{indicacion}{
\begin{enumerate}
\item Editar el gráfico realizado anteriormente haciendo un doble
click sobre él.

\item Seleccionar los puntos haciendo click sobre alguno de ellos.
\item Seleccionar el menú \menu{Gráfico->Añadir elemento de
gráfico->Linea de ajuste total} (También se podría usar en lugar
del menu, la barra de herramientas) \item Cerrar el editor de
gráficos, cerrando la ventana.
\end{enumerate}}
\end{indicacion}


\item Calcular la recta de regresión de \textsf{A} sobre
\textsf{B} y dibujarla sobre el correspondiente diagrama de
dispersión.

\begin{indicacion}{
Repetir los pasos de los apartados anteriores pero escogiendo como
variable \texttt{Dependiente} la variable \textsf{A}, y como
variable \texttt{Independiente} la variable \textsf{B}}
\end{indicacion}


\item ¿Son grandes los residuos? Comentar los resultados.
\end{enumerate}


\item  En una licenciatura se quiere estudiar la relación entre el número
medio de horas de estudio diarias y el número de asignaturas suspensas. Para ello se
obtuvo la siguiente muestra:
\begin{center}
\begin{tabular}{cccccccc}
  Horas & Suspensos &  & Horas & Suspensos & & Horas & Suspensos  \\
  \cline{1-2}\cline{4-5}\cline{7-8}
  3.5 & 1 & & 2.2 & 2 & & 1.3 & 4 \\
  0.6 & 5 & & 3.3 & 0 & & 3.1 & 0 \\
  2.8 & 1 & & 1.7 & 3 & & 2.3 & 2 \\
  2.5 & 3 & & 1.1 & 3 & & 3.2 & 2 \\
  2.6 & 1 & & 2.0 & 3 & & 0.9 & 4 \\
  3.9 & 0 & & 3.5 & 0 & & 1.7 & 2 \\
  1.5 & 3 & & 2.1 & 2 & & 0.2 & 5 \\
  0.7 & 3 & & 1.8 & 2 & & 2.9 & 1 \\
  3.6 & 1 & & 1.1 & 4 & & 1.0 & 3 \\
  3.7 & 1 & & 0.7 & 4 & & 2.3 & 2 \\
\end{tabular}

\end{center}

Se pide:

\begin{enumerate}
\item  Crear las variables \textsf{horasestudio} y
\textsf{suspensos} e introducir estos datos.

\item  Calcular la recta de regresión de \textsf{suspensos} sobre
\textsf{horasestudio} y dibujarla.

\begin{indicacion}{
\begin{enumerate}
\item Seleccionar el menú \menu{Analizar->Regresion->Lineal...}.

\item Seleccionar la variable \textsf{suspensos} en el campo
\texttt{Dependiente} del cuadro de diálogo.

\item Seleccionar la variable \textsf{horasestudio} en el campo
\texttt{Independiente} del cuadro de diálogo y hacer click sobre
el botón \texttt{Aceptar}.

\item Para escribir la recta, observaremos en la ventana de
resultados obtenida, la tabla denominada \texttt{Coeficientes}, y
en la columna \texttt{B} de los \texttt{Coeficientes no
estandarizados}, encontramos en la primera fila la
\textsf{constante} de la recta y en la segunda la
\texttt{pendiente}.

\item Seleccionar el menú \menu{Gráficos->Dispersión...}, elegir
la opción \texttt{simple} y  hacer click sobre el botón
\texttt{Definir}. \item Seleccionar la variable \textsf{suspensos}
en el campo \texttt{Eje Y} del cuadro de diálogo. \item
Seleccionar la variable \textsf{horasestudio} en el campo
\texttt{Eje X} del cuadro de diálogo y hacer click sobre el botón
\texttt{Aceptar}.

\item Editar el gráfico realizado anteriormente haciendo un doble
click sobre él.

\item Seleccionar los puntos haciendo click sobre alguno de ellos.
\item Seleccionar el menú \menu{Gráfico->Añadir elemento de
gráfico->Linea de ajuste total} (También se podría usar en lugar
del menu, la barra de herramientas) \item Cerrar el editor de
gráficos, cerrando la ventana.

\end{enumerate}}
\end{indicacion}

\item Indicar el coeficiente de regresión de \textsf{suspensos}
sobre \textsf{horasestudio}. ¿Cómo lo interpretarías?

\item La relación lineal entre estas dos variables, ¿es mejor o peor que la del
ejercicio anterior? Comentar los resultados.
\end{enumerate}
\end{enumerate}

\section{Problemas}
\begin{enumerate}[leftmargin=*]
\item  Se determina la pérdida de actividad que experimenta un
medicamento desde el momento de su fabricación a lo largo del tiempo, obteniéndose
el siguiente resultado:

\begin{center}
\begin{tabular}{|c|c|c|c|c|c|}
\hline Tiempo (en años) & 1 & 2 & 3 & 4 & 5 \\ \hline Actividad restante (\%) & 96 &
84 & 70 & 58 & 52 \\ \hline
\end{tabular}
\end{center}

Se desea calcular:

\begin{enumerate}
\item  La relación fundamental (recta de regresión) entre
actividad restante y tiempo transcurrido.

\item ¿En qué porcentaje disminuye la actividad cada año que pasa?
\end{enumerate}

\item Al realizar un estudio sobre la dosificación de un cierto
medicamento, se trataron 6 pacientes con dosis diarias de 2 mg, 7 pacientes con 3 mg
y otros 7 pacientes con 4 mg. De los pacientes tratados con 2 mg, 2 curaron al cabo
de 5 días, y 4 al cabo de 6 días. De los pacientes tratados con 3 mg diarios, 2
curaron al cabo de 3 días, 4 al cabo de 5 días y 1 al cabo de 6 días. Y de los
pacientes tratados con 4 mg diarios, 5 curaron al cabo de 3 días y 2 al cabo de 5
días. Se pide:

\begin{enumerate}
\item  Calcular la recta de regresión del tiempo de curación con respecto a la dosis
suministrada.

\item  Calcular los coeficientes de regresión. Interpretar los resultados.
\end{enumerate}

\end{enumerate}
