% Version control information:
%$HeadURL: https://practicas-spss.googlecode.com/svn/trunk/regresion_no_lineal/regresion_no_lineal.tex $
%$LastChangedDate: 2010-09-27 16:37:11 +0200 (lun, 27 sep 2010) $
%$LastChangedRevision: 3 $
%$LastChangedBy: asalber $
%$Id: regresion_no_lineal.tex 3 2010-09-27 14:37:11Z asalber $

\chapter{Regresión No Lineal}

\section{Fundamentos teóricos}
La regresión simple tiene por objeto la construcción de un modelo
funcional $y=f(x)$ que explique lo mejor posible la relación entre
dos variables $Y$ (variable dependiente) y $X$ (variable
independiente) medidas en una misma muestra.

Ya vimos que, dependiendo de la forma de esta función, existen muchos tipos de regresión simple. Entre los más habituales están:
\begin{center}
\begin{tabular}{|l|c|}
\hline
 Familia de curvas       &     Ecuación genérica      \\
\hline\hline
 Lineal                  &          $y=b_0+b_1x$          \\
\hline
 Cuadrática              &       $y=b_0+b_1x+b_2x^2$        \\
\hline
 Cúbica & $y=b_0+b_1x+b_2x^2+b_3x^3$ \\
\hline
 Potencia               &       $y=b_0\cdot x^{b_1}$       \\
\hline
 Exponencial             &     $y=b_0\cdot e^{b_1x}$      \\
\hline
 Logarítmica             &       $y=b_0+b_1\ln x$        \\
\hline
Inversa             &       $y=b_0+\frac{b_1}{x}$        \\
\hline
Compuesto              &       $y=b_0b_1^x$        \\
\hline
Crecimiento             &       $y= e^{b_0 + b_1x}$        \\
\hline
G (Curva-S)             &       $y= e^{b_0 +\frac{b_1}{x} }$        \\
\hline
\end{tabular}
\end{center}

La elección de un tipo de modelo u otro suele hacerse según la forma
de la nube de puntos del diagrama de dispersión. A veces estará
claro qué tipo de modelo se debe construir, tal y como ocurre en los
diagramas de dispersión de la figura~\ref{g:tiposrelaciones2}. Pero
otras veces no estará tan claro, y en estas ocasiones, lo normal es
ajustar los dos o tres modelos que nos parezcan más convincentes,
para luego quedarnos con el que mejor explique la relación entre $Y$
y $X$, mirando el coeficiente de determinación\footnote{Ver la
práctica de correlación.} de cada modelo.

\begin{figure}[h!]
\centering 
\subfigure[Sin relación.]{\scalebox{0.5}{\input{regresion_lineal_simple/img/diagrama_dispersion_sin_relacion}}}\qquad
\subfigure[Relación lineal.]{\scalebox{0.5}{\input{regresion_lineal_simple/img/diagrama_dispersion_relacion_lineal}}}\qquad
\subfigure[Relación polinómica.]{\scalebox{0.5}{\input{regresion_lineal_simple/img/diagrama_dispersion_relacion_parabolica}}}\\
\subfigure[Relación exponencial.]{\scalebox{0.5}{\input{regresion_lineal_simple/img/diagrama_dispersion_relacion_exponencial}}}\qquad
\subfigure[Relación logarítmica.]{\scalebox{0.5}{\input{regresion_lineal_simple/img/diagrama_dispersion_relacion_logaritmica}}}\qquad
\subfigure[Relación inversa.]{\scalebox{0.5}{\input{regresion_lineal_simple/img/diagrama_dispersion_relacion_inversa}}}\\
\caption{Diagramas de dispersión correspondientes a distintos tipos de relaciones
entre variables.} \label{g:tiposrelaciones2}
\end{figure}

Ya vimos en la práctica sobre regresión lineal simple, cómo
construir rectas de regresión. En el caso de que optemos por ajustar
un modelo no lineal, la construcción del mismo puede realizarse
siguiendo los mismos pasos que en el caso lineal. Básicamente se
trata de determinar los parámetros del modelo que minimizan la suma
de los cuadrados de los residuos en $Y$. En los modelos
potencia y exponencial, el sistema aplica transformaciones
logarítmicas a las variables y después ajusta un modelo lineal a los
datos transformados. En el modelo inverso, el sistema sustituye la
variable dependiente por su inverso antes de estimar la ecuación
de regresión.

\clearpage
\newpage


\section{Ejercicios resueltos}
\begin{enumerate}[leftmargin=*]
\item En un experimento se ha medido el número de bacterias por unidad de volumen en un cultivo, cada hora transcurrida, obteniendo los siguientes resultados:
\begin{center}
\begin{tabular}{c|ccccccccc}
Horas & 0 & 1 & 2 & 3 & 4 & 5 & 6 & 7 & 8  \\
\hline
Nº Bacterias & 25 & 32 & 47 & 65 & 92 & 132 & 190 & 275 & 362
\end{tabular}
\end{center}

Se pide:
\begin{enumerate}
\item  Crear las variables \variable{horas} y \variable{bacterias} e introducir estos datos.

\item  Dibujar el diagrama de dispersión correspondiente. En vista del diagrama, ¿qué tipo de modelo crees que
explicará mejor la relación entre el número de bacterias y el tiempo transcurrido?
\begin{indicacion}
\begin{enumerate}
\item Seleccionar el menú \menu{Gráficos\flecha Cuadros de diálogo antiguos\flecha 
Dispersión/Puntos...}, elegir la opción \opcion{Dispersión simple} y  
hacer click sobre el botón \boton{Definir}.
\item Seleccionar la variable \variable{bacterias} en el 
campo \campo{Eje Y} del cuadro de diálogo. 
\item Seleccionar la variable \variable{horas} en el campo \campo{Eje X} 
del cuadro de diálogo y hacer click sobre el botón \boton{Aceptar}.
\end{enumerate}
\end{indicacion}

\item Hacer una comparativa de los distintos modelos de regresión en función del coeficiente de determinación. ¿Qué
tipo de modelo es el mejor?
\begin{indicacion}
\begin{enumerate}
\item Seleccionar el menú \menu{Analizar\flecha Regresión\flecha Estimación curvilínea...}.
\item Seleccionar la variable \variable{bacterias} en el campo \campo{Dependientes} del cuadro de diálogo.
\item Seleccionar la variable \variable{horas} en el campo  \campo{Independiente/variable} del cuadro de diálogo.
\item Desmarcar la opción \opcion{Representar los modelos}.
\item Marcar las opciones lineal, cuadrático, cúbico, exponencial y logarítmico, y hacer click sobre el botón \boton{Aceptar}.
\end{enumerate}
\end{indicacion}
\item En  vista de lo anterior, calcular el modelo de regresión que 
mejor explique la relación entre \variable{bacterias} y \variable{horas}.
\begin{indicacion}Utilizar los coeficientes que aparecen en el punto 
anterior y la tabla de la parte de fundamentos teóricos.
\end{indicacion}

\item Según el modelo anterior, ¿cuántas bacterias habrá al cabo de 3 
horas y media del inicio del cultivo? ¿Y al cabo
de 10 horas? ¿Son fiables estas predicciones?
\begin{indicacion}
\begin{enumerate}
\item Crear una nueva variable \variable{valores} e introducir 
los valores de las horas para los que queremos predecir las bacterias.
\item Seleccionar el menú \menu{Transformar\flecha Calcular variable...}.
\item Introducir el nombre de la nueva variable 
\variable{prediccion} en el campo \campo{Variable de destino}
del cuadro de diálogo.
\item Introducir la ecuación del mejor modelo en el campo 
\campo{Expresión numérica}, utilizando los coeficientes obtenidos 
anteriormente y la variable \variable{valores} y hacer click sobre 
el botón \boton{Aceptar}.
\end{enumerate}
\end{indicacion}

\item Dar una predicción lo más fiable posible del tiempo que tendría 
que transcurrir para que en el cultivo hubiese 100 bacterias.
\begin{indicacion}
Repetir los pasos del apartado anterior introduciendo la variable 
\variable{horas} en el campo \campo{Dependientes} y
la variable \variable{bacterias} en el campo 
\campo{Independiente/Variables}.
\end{indicacion}
\end{enumerate}

\item Se han medido dos variables $S$ y $T$ en 10 individuos, 
obteniéndose los siguientes resultados:
\begin{center}
$(-1.5 \,,\, 2.25)$\,,\, $(0.8 \,,\, 0.64)$\,,\, $(-0.2 \,,\, 0.04)$\,,\, $(-0.8 \,,\, 0.64)$\,,\, $(0.4 \,,\, 0.16)$\,,\,\\
$(0.2 \,,\, 0.04)$\,,\, $(-2.1 \,,\, 4.41)$\,,\, $(-0.4 \,,\, 0.16)$\,,\, $(1.5 \,,\, 2.25)$\,,\, $(2.1 \,,\, 4.41)$.
\end{center}
Se pide:
\begin{enumerate}
\item  Crear las variables \variable{S} y \variable{T} e introducir estos datos.
\item Calcular la recta de regresión de \variable{T} sobre \variable{S}. 
Dibujar dicha recta sobre el diagrama de dispersión. ¿Podemos afirmar 
que $S$ y $T$ son independientes?
\begin{indicacion}
\begin{enumerate}
\item Seleccionar el menú \menu{Analizar\flecha Regresión\flecha Lineales...}.
\item Seleccionar la variable \variable{T} en el campo 
\variable{Dependientes} del cuadro de diálogo.
\item Seleccionar la variable \variable{S} en el campo 
\campo{Independientes} del cuadro de diálogo y hacer click sobre
el botón \boton{Aceptar}.
\item Para escribir la ecuación de la recta, observaremos en la ventana 
de resultados obtenida, la tabla denominada
\resultado{Coeficientes}, y en la columna \resultado{B} de los 
\resultado{Coeficientes no estandarizados}, encontramos en la primera 
fila la \resultado{constante} de la recta y en la segunda 
la \resultado{pendiente}.
\item Seleccionar el menú \menu{Gráficos\flecha Cuadros de diálogo antiguos\flecha 
Dispersión/Puntos...}, elegir la opción \opcion{Dispersión simple} y  
hacer click sobre el botón \boton{Definir}.
\item Seleccionar la variable \variable{T} en el campo \campo{Eje Y} 
del cuadro de diálogo. 
\item Seleccionar la variable \variable{S} en el campo \campo{Eje X} 
del cuadro de diálogo y hacer click sobre el botón \boton{Aceptar}.
\item Editar el gráfico realizado anteriormente haciendo un doble 
click sobre él.
\item Seleccionar los puntos haciendo click sobre alguno de ellos.
\item Seleccionar el menú \menu{Elementos\flecha Linea de ajuste total} 
(También se podría usar en lugar del menu, la barra de herramientas) 
\item Cerrar la ventana \texttt{Propiedades}.
\item Cerrar el editor de gráficos, cerrando la ventana.
\end{enumerate}
\end{indicacion}

\item Hacer una comparativa de los distintos modelos de regresión en 
función del coeficiente de determinación. ¿Qué
tipo de relación existe entre $T$ y $S$?
\begin{indicacion}
\begin{enumerate}
\item Seleccionar el menú \menu{Analizar\flecha Regresión\flecha Estimación curvilínea...}.
\item Seleccionar la variable \variable{T} en el campo \campo{Dependientes} 
del cuadro de diálogo.
\item Seleccionar la variable \variable{S} en el campo 
\texttt{Independiente/variable} del cuadro de diálogo.
\item Desmarcar la opción \opcion{Representar los modelos}.
\item Marcar las opciones lineal, cuadrático, cúbico y exponencial y hacer click sobre el botón \boton{Aceptar}.
\end{enumerate}
\end{indicacion}

\item En vista de lo anterior, ajustar el modelo de regresión más apropiado.
\begin{indicacion}
Utilizar los coeficientes que aparecen en el punto anterior y la tabla 
de la parte de fundamentos teóricos.
\end{indicacion}
\end{enumerate}

\end{enumerate}


\section{Ejercicios propuestos}
\begin{enumerate}[leftmargin=*]
\item En un centro dietético se está probando una nueva dieta de adelgazamiento en una
muestra de 12 individuos. Para cada uno de ellos se ha medido el número de días que
lleva con la dieta y el número de kilos perdidos desde entonces, obteniéndose los
siguientes resultados:
\begin{center}
(33 , 3.9), (51 , 5.9), (30 , 3.2), (55 , 6.0), (38 , 4.9), (62 , 6.2),\\
(35 , 4.5), (60 , 6.1), (44 , 5.6), (69 , 6.2), (47 , 5.8), (40 , 5.3)
\end{center}
Se pide:
\begin{enumerate}
  \item Dibujar el diagrama de dispersión. Según la nube de puntos, ¿qué tipo de
  modelo explicaría mejor la relación entre los kilos perdidos y los días de dieta?
  \item Construir el modelo de regresión que mejor explique la relación entre los kilos perdidos  y los días de dieta.
  \item Utilizar el modelo construido para predecir el número de kilos perdidos tras 40
  días de dieta y tras 100 días. ¿Son fiables estas predicciones?
\end{enumerate}

\item La concentración de un fármaco en sangre, $C$ en mg/dl, es función del tiempo, $t$ en horas, y viene dada por la
siguiente tabla: 
\[
\begin{array}{|l|r|r|r|r|r|r|r|}
\hline
\text{t} & 2 & 3 & 4 & 5 & 6 & 7 & 8\\
\hline
\text{C} & 25 & 36 & 48 & 64 & 86 & 114 & 168\\
\hline
\end{array}
\]
Se pide: 
\begin{enumerate}
\item Según el modelo exponencial, ¿qué concentración de fármaco habría a las $4.8$ horas? ¿Es fiable la predicción?
Justificar adecuadamente la respuesta.
\item Según el modelo logarítmico, ¿qué tiempo debe pasar para que la concentración sea de 100 mg/dl?
\end{enumerate}

\end{enumerate}